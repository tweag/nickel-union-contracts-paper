\documentclass[sigplan,10pt,review,anonymous]{acmart}
%\usepackage{amssymb}
\usepackage{mathtools,amsmath,amsfonts,amsxtra,amsthm,mathrsfs,mathpartir}
\usepackage{listings}
\usepackage{xcolor}
\usepackage{array}
\usepackage[toc,page]{appendix}
\usepackage{graphicx}
\usepackage{varioref}
\usepackage{verbatim,comment}
\usepackage{longtable}
\usepackage{tikz}
\usepackage{url}
\usepackage{hyperref}

\definecolor{OliveGreen}{RGB}{128,128,0}
%%%%%%%%%%%%%%%%% Editing marks %%%%%%%%%%%%%%%%%

  % TOGGLE ME to turn off all the commentary:
  \InputIfFileExists{no-editing-marks}{
    \def\noeditingmarks{}
  }

  \usepackage{xargs}
  \usepackage[colorinlistoftodos,prependcaption,textsize=tiny]{todonotes}
  % ^^ Need for pgfsyspdfmark apparently?
  \ifx\noeditingmarks\undefined
      \setlength{\marginparwidth}{1.2cm} % A size that matches the new PACMPL format
      \newcommand{\Red}[1]{{\color{red}{#1}}}
      \newcommand{\newaudit}[1]{{\color{blue}{#1}}}
      \newcommand{\note}[1]{{\color{blue}{\begin{itemize} \item {#1} \end{itemize}}}}
      \newenvironment{alt}{\color{red}}{}

      \newcommandx{\unsure}[2][1=]{\todo[linecolor=red,backgroundcolor=red!25,bordercolor=red,#1]{#2}}
      \newcommandx{\info}[2][1=]{\todo[linecolor=green,backgroundcolor=green!25,bordercolor=green,#1]{#2}}
      \newcommandx{\change}[2][1=]{\todo[linecolor=blue,backgroundcolor=blue!25,bordercolor=blue,#1]{#2}}
      \newcommandx{\inconsistent}[2][1=]{\todo[linecolor=blue,backgroundcolor=blue!25,bordercolor=red,#1]{#2}}
      \newcommandx{\critical}[2][1=]{\todo[linecolor=blue,backgroundcolor=blue!25,bordercolor=red,#1]{#2}}
      \newcommand{\improvement}[1]{\todo[linecolor=pink,backgroundcolor=pink!25,bordercolor=pink]{#1}}
      \newcommandx{\resolved}[2][1=]{\todo[linecolor=OliveGreen,backgroundcolor=OliveGreen!25,bordercolor=OliveGreen,#1]{#2}} % use this to mark a resolved question
  \else
  %    \newcommand{\Red}[1]{#1}
      \newcommand{\Red}[1]{{\color{red}{#1}}}
      \newcommand{\newaudit}[1]{#1}
      \newcommand{\note}[1]{}
      \newenvironment{alt}{}{}
  %    \renewcommand\todo[2]{}
      \newcommand{\unsure}[2][1=]{}
      \newcommand{\info}[2][1=]{}
      \newcommand{\change}[2]{}
      \newcommand{\inconsistent}[2]{}
      \newcommand{\critical}[2]{}
      \newcommand{\improvement}[1]{}
      \newcommand{\resolved}[2]{}
  \fi

%%%%%%%%%%%%%%%%% /Editing marks %%%%%%%%%%%%%%%%%

%%%%%%%%%%% LST listing confinguraiton %%%%%%%%%%%

\lstdefinelanguage{nickel}{
    keywords=[1]{
      if,
      then,
      else,
      switch,
    },
    keywords=[2]{
      let,
      rec,
      fun,
      in
    },
    keywords=[3]{
      Num,
      Bool,
      Str,
      List,
      Dyn,
      forall,
    },
    keywordsprefix=\#,
    keywords=[4]{
      true,
      false,
      null
    },
    keywords=[5]{
      doc,
      default,
    },
    sensitive=true, % keywords are case-sensitive
    morecomment=[l]{//}, % l is for line comment
    morecomment=**[is][\color{gray}]{\$}{\$},
    morestring=[b]", % defines that strings are enclosed in double quotes
    morestring=[s]{m\#"}{"\#m},
    moredelim=[is][\color{gray}]{\$}{\$},
    literate=
        *{->}{$\rightarrow$}1
        {=>}{$\Rightarrow$}1
        {>=}{$\geq$}1
        {<=}{$\leq$}1
} %

% Copied from https://hal.inria.fr/hal-01140459/file/racket.inc.tex
\lstdefinelanguage{racket} {
  morekeywords=[1]{define, define-syntax, define-macro, lambda, define-stream, stream-lambda},
  morekeywords=[2]{begin, call-with-current-continuation, call/cc,
    call-with-input-file, call-with-output-file, case, cond,
    do, else, for-each, if,
    let*, let, let-syntax, letrec, letrec-syntax,
    define-context, define-controller, Integer, Boolean, get, when-required, when-provided,
    maybe_publish, require, submod, or/c, ->, \#\%module-begin,
    always_publish, with-syntax, define-struct/contract, syntax-case,
    define/contract,
    let-values, let*-values,
    module, provide,
    and, or, not, delay, force,
    \#`, \#',
    \#lang, implement, begin-for-syntax, rename-out,
    quasiquote, quote, unquote, unquote-splicing,
    map, fold, syntax, syntax-rules, eval, environment, query },
  morekeywords=[3]{import, export},
  alsoletter={',`,-,/,>,<,\#,\%},
  morecomment=[l]{;},
%  literate={lambda}{{\lambdaup}}1, % lambda -- look at https://tex.stackexchange.com/questions/119879/math-symbols-in-tt-font
  moredelim=**[is][\color{light-gray}]{<<@<<}{>>@>>},
  moredelim=**[is][\itshape\color{OliveGreen}]{<<;<<}{>>;>>},
  morecomment=[s]{\#|}{|\#},
  sensitive=true,
}

\lstset{
  language={nickel},
  basicstyle=\small\ttfamily, % Global Code Style
  captionpos=b, % Position of the Caption (t for top, b for bottom)
  extendedchars=true, % Allows 256 instead of 128 ASCII characters
  tabsize=2, % number of spaces indented when discovering a tab
  columns=fixed, % make all characters equal width
  keepspaces=true, % does not ignore spaces to fit width, convert tabs to spaces
  showstringspaces=false, % lets spaces in strings appear as real spaces
  breaklines=true, % wrap lines if they don't fit
  frame=trbl, % draw a frame at the top, right, left and bottom of the listing
  frameround=tttt, % make the frame round at all four corners
  framesep=4pt, % quarter circle size of the round corners
  numbers=left, % show line numbers at the left
  numberstyle=\tiny\ttfamily, % style of the line numbers
  escapeinside={(*@}{@*)} % escape sequence to insert latex
  % commentstyle=\color{dark-grey}, % style of comments
  % keywordstyle=[1]\color{blue-portage}, % style of keywords
  % keywordstyle=[2]\color{orange-vivid-tangerine}, % style of keywords
  % keywordstyle=[3]\color{blue-portage}, % style of keywords
  % keywordstyle=[4]\color{green-pea}, % style of keywords
  % keywordstyle=[5]\color{pink-froly}, % style of keywords
  % stringstyle=\color{blue-marguerite}, % style of strings
}

\newcommand{\nickel}[1]{\lstinline[language=nickel]{#1}}
\newcommand{\racket}[1]{\lstinline[language=racket]{#1}}

\bibliographystyle{ACM-Reference-Format}

\begin{abstract}
This paper reports an attempt to incorporate union and intersection types in
Nickel, a configuration programming language with a gradual type system and
first-class contracts. While the end result looks appealing, it turns out the
concrete impact on both the design and the implementation of the language is
unexpectedly high. We review the issues raised in particular by the
implementation of contracts for unions and intersections, and why we think the
cost ends up too high for the benefit. We conclude by suggesting different leads
to represent and type unions.
\end{abstract}


\title{Contracts for unions and intersections are hard for Nickel}
\author{Teodoro Freund}
\affiliation{
  \institution{???}
  \city{Buenos Aires}
  \country{Argentina}
}
\email{???}
\author{Yann Hamdaoui}
\affiliation{
  \institution{Tweag}
  \city{Paris}
  \country{France}
}
\email{yann.hamdaoui@tweag.io}
\author{Arnaud Spiwack}
\affiliation{
  \institution{Tweag}
  \city{Paris}
  \country{France}
}
\email{arnaud.spiwack@tweag.io}

\begin{document}

\maketitle

\unsure{TODO: CCS classification; Keywords}
\section{Introduction}
\info{Goals of the paper: position paper: unions/intersections nice in
theory problematic in practice. In particular non-orthogonal with the
rest of the features of your language.}

Build systems, package managers, operating systems, cloud infrastructure,
continuous integration and web services\resolved{Also continuous integration}
are examples of modern complex software systems which require an extensive and
non-trivial configuration. To handle this complexity, the declarative approach
has become more and more prominent: instead of applying a sequence of
transformations to a system in order to reach a desired state, this state is
described as static textual data. Some tool is then in charge of applying the
necessary transformations. This approach makes the configuration persistent,
reproducible, versionable and subject to static analysis. Text-based
configurations are thus led to play an important role in critical aspects of
software engineering, including security, availability, and maintainability.

However, static text-based configurations show their limits beyond a certain
size and complexity. Data cannot be transformed, combined, shared, nor validated
(without resorting to separate schemas and tools).  This lack of expressivity
results at best in boilerplate and duplication of information, but it can also
make data invalid or inconsistent.  Correctly configuring a modern system is ard
and failures may have substantial negative consequences.

This is why the authors have been developing the Nickel
language\cite{NickelRepo}. Nickel is a configuration programming language,
meaning that a Nickel program evaluates to a value that must be
straightforwardly serializable to JSON, YAML or alike. Nickel aims at being
lightweight and easy to write while providing support for writing correct code
and generating valid data. Nickel originated as a rethink of the language of the
Nix package manager, used to describe one of the biggest sets of packaged
software in existence\cite{repology}.  Nickel targets build systems, cloud
deployment tools (Kubernetes, Terraform or NixOps), or anything requiring a
non-trivial configuration as well.

\subsection*{To type or not to type}

The end goal being data generation, Nickel is a dynamic, pure and functional
language. Our goal is that writing a simple configuration in Nickel feels as
easy as writing JSON or YAML. Unlike JSON, though, Nickel anticipates large
configurations by being programmable, and by providing means of ensuring
both correctness and validity: 

\begin{description}
    \item[(correctness)] The program runs without errors. Errors may be type
        errors, raised for example when evaluating the nonsensical division
        \nickel{1 / "ftp"}. More generally, they can any the violation of a pre-
        or post-condition. When such an error cannot be prevented, we want at
        least early failures and relevant error messages.
    \item[(validity)] The program generates valid data, i.e. data that
        conforms to a schema (imposed by the system consuming the
        configuration). For example, when configuring a systemd service, the
        schema may require the presence of a field \lstinline+port+ that holds a
        valid port number.
\end{description}

A traditional answer is to use static types, in particular static types for
correctness. However, Nickel programs are of a special kind: they are usually
simple, terminating programs that run on fixed inputs. Correctness errors will
show up at evaluation anyway, or will not matter (dead code). With respect to
validation, traditional types seem too limited. Validation commonly involves
checking requirements such as being a valid port number or a valid URL which
are, while not technically impossible, really hard to check statically.

In a dynamic language, it is natural and easier to validate data at runtime.
This underlies the approach of dynamic typing. But dynamically typed languages
usually only check the same simple predicates as static types (being a number,
being a string, etc.), and solely for data that are arguments to a primitive
operation. For validation, we want to check arbitrary predicates on arbitrary
values. This is provided in a basic form by assertions, available in most
languages out there. An assertion is an expression that verifies that a
predicate holds, or abort the program otherwise.  While assertions are fine for
simple sanity checks, they can quickly becomes verbose, clumsy to compose and to
reuse, and provide very limited context attached to an error. It turns out there
is a better tool than plain assertions: contracts.

\subsection*{Contracts}

Assertions are usually mainly used for correctness by enforcing pre- and
post-conditions of functions at runtime. In their foundational
paper\cite{FindlerFelleisenHOContracts}, Findler and Felleisen introduce a
principled approach to run-time assertion checking called contracts. As opposed
to plain assertions, contracts can be composed naturally into more complex ones.
Nickel features such a contract system. For example, contracts nicely supports
higher-order functions. If the user has written an \nickel{Even} contract for
number, they can use it right away to form a \nickel{#Even -> #Int} contract, as
illustrated in Figure~\ref{fig:contract-higher-order}. This contracts will check
at each call that the argument is an even number (pre-condition) and that the
return value is a integer (post-condition). The \nickel{#} character
distinguishes contracts in Nickel. Contracts also introduce the notion of blame
which is crucial to good error reporting in presence of higher-order contracts
and other composed contracts.

\begin{figure}
  \begin{center}
\begin{lstlisting}[language=nickel]
let Even = contracts.fromPred (fun val =>
  builtins.isNum val
  && val % 2 == 0) in
let Int = contracts.fromPred (fun val =>
  builtins.isNum val
  && val % 1 == 0) in
{divBy2 | #Even -> #Int = fun x => x/2}
\end{lstlisting}
  \end{center}
\caption{Higher-order contract}
\label{fig:contract-higher-order}
\end{figure}

In our opinion, one of the main appeal of Nickel is this ability to write a
simple validation function for, say, a port number, and have it immediately
available as a contract that can be used to construct a other contracts by
writing natural type-like expressions. Another example is given in
Figure~\ref{fig:contract-composition}. The first line defines a port contract
from a predicate and use it in the definition of a larger
\lstinline+Configuration+ record contract. The \lstinline+Configuration+
contract checks that a value is a record with the following shape: it must have
a \lstinline+port+ field obeying the \lstinline+Port+ contract, and a
\lstinline+host+ field obeying a \lstinline+Host+ contract. It plays the role of
a \emph{schema} used to validate the final data, which is exactly what we were
looking for in the previous section. Not only it plays the role of a schema, but
it also \emph{looks} like one, as opposed to a less readable plain validation
function. The \lstinline+|+ operator entails that a value respect a contract,
akin to \lstinline+:+ for types. We assume that a \lstinline+Host+ contract has
been defined in a similar way. \unsure{What's missing in this section is an
example of how contracts differ from standard dynamic type checking on some
function. Namely, the quality of error messages.}

\begin{figure}
  \begin{center}
\begin{lstlisting}[language=nickel]
let Port = contracts.fromPred (fun value =>
  builtins.isNum value && value % 1 == 0
  && value >= 0 && value <= 65353) in
let Configuration = {
  host | #Host,
  port | #Port,
} in
{host = "nickel-lang.org", port = 80}
| #Configuration
\end{lstlisting}
  \end{center}
\caption{Contract composition}
\label{fig:contract-composition}
\end{figure}

Contracts have another perk: they correspond to the type casts of gradual
typing, a discipline for mixing static and dynamic typing in the same language.
As such, they are a precious ingredient for handling the interaction of
statically typed and dynamically typed code. While we focused on the writing of
configuration, Nickel is a proper programming language that is also designed to
write generic and reusable library code. There, the typing trade-off is
different, and static typing is more adapted. Nickel has a gradual type system,
which is yet another argument for having contracts. Let us end this section by
stressing the following points:
\begin{itemize}
    \item \textbf{Even in a purely dynamically typed language, contracts are an
        appealing feature for validation and correctness}. The gradual
        typing part deserved to be mentioned, but is is out of the scope of this
        paper. 
    \item \textbf{User-defined contracts are essential for validation}. Some contract
        systems (as the one of~\ref{RootCauseOfBlame} that will be extensively
        discussed later in the paper) do not feature user-defined contracts.
        Rather, a fixed set of built-in contracts such as \nickel{Number} or
        \nickel{String} are provided, very much like base types. The absence of
        user-defined contracts is understandable when using contracts as a
        dynamic typing discipline for correctness. But a system without
        user-defined contracts is limited in the same way as static typing with
        respect to validation.
\end{itemize}

\newpage

\section{Features of languages}
\info{Zoology of various features that we will eventually show
  conflict with this or that property or implementation of union and
  intersection. Including user-define contracts.}

Programming language research no longer lives on the abstract world
of ideal surfaces.
Every addition needs to be compatible with years of research
and development, and this is hard.
Success was unavoidable.

In this section we briefly introduce some aspects of real world
programming languages that will cause trouble while interacting
with union and intersection dynamic checks.

\subsection*{Flat contracts}
\label{sec:flat-contracts}

\unsure{Should we define contracts prior to this?}
User defined contracts, also called flat contracts, are a particular
kind of contract that behaves similar to an assertion, and whose
main point is to apply some arbitrary predicate (as arbitrary as the
host language allows it) to a value; if the predicate evaluates to
true, the contract passes, otherwise it fails.

For instance, we could define a flat contract \nickel{even} as
\nickel{fun x => x \% 2 == 0}, that checks
whether a number is even by checking the remainder of dividing it by 2.
And then use it to check that a function always returns an
even number, when applied to an \nickel{Integer}, as shown in
Figure \ref{fig:num-to-even}.
Or even, we could come up with a contract that checks that a function
maps \nickel{true} to \nickel{false} and \nickel{false} to \nickel{true},
and use it as a contract for the \nickel{not} function, as can be seen on Figure
\ref{fig:checked-not}.

\begin{figure}[h]
\begin{center}
\begin{lstlisting}[language=nickel]
let f = (fun n => n * 2) @ (Integer -> even)
\end{lstlisting}
  \end{center}
\caption{A function wrapped with a contract stating that it always returns
an even number}
\label{fig:num-to-even}
\end{figure}

% $$let~f~=~(\lambda n. ~ n * 2)@(Number ~ \rightarrow ~ even?)$$

\begin{figure}[h]
  \begin{center}
\begin{lstlisting}[language=nickel]
let notC = fun f => (f true == false) and (f false == true) in
let checkedNot = not @ notC
\end{lstlisting}
\end{center}
\caption{A contract that checks the not function behaves as expected}
\label{fig:checked-not}
\end{figure}

This type of contracts are present on many different languages,
for instance, the Eiffel programming language\unsure{Citation?}, the precursor
of the Design by Contract ideals, allows to assert
these kinds of expression as pre- and post-conditions on
functions\cite{EiffelDesignByContract}.
A somewhat newer language, the D programming
language, allows, among other things, to have invariants on
classes\unsure{I (Arnaud) believe Eiffel has this feature as well,
  doesn't it?},
written with flat contracts, that are
dinamically checked at the frontiers of every public method \cite{DLangContracts}.

Finally, the Racket programming language also has a system to work with
contracts, powerful enough to define flat contracts
\footnote{And a gradual typing system
built on top of that, but that goes beyond the scope of this
work.}
, and
to compose them with other kinds of dynamic checks,
like higher order contracts or a lightweight take on union
and intersection contracts\cite{RacketContracts}.
\todo{Talk about this, probably later on the paper.
See Related Works section}

Even if flat contracts may seem like too permissive, the ability to make
non trivial checks on values (that is, to validate their output) is expected
by programmers on dynamic languages, where testing and runtime
assertions are extremely useful tools for safe programming, and
programmers are accustomed to express complex predicates
over values.


\subsection*{Code manipulation (optimizations)}
\label{sec:optimizations}
\info{Present inlining and CSE as two major code
optimizations.
Purity and immutability as two important factors.}

The performance of modern programs heavily relies on the optimizations performed
by the compiler or the interpreter. Even more so for functional languages, whose
execution model is often far removed from actual processors, causing naive
execution to exhibit unacceptable slowdowns.

One crucial optimization is
inlining (Figure \ref{fig:optimizations-inlining-ex}). Functional programs tend
to make heavy use of functions, and a function call is a costly operation from
the point of view of low-level execution. Inlining eliminates a function
application by directly substituting the function for its definition at compile
time (or before execution, for interpreted language). This is especially
efficient for small functions that are called repeatedly.

\info{I think it's better to use [h] on figures, we can discuss it}
\begin{figure}[h]
  \begin{center}
\begin{lstlisting}[language=nickel,title={Source program}]
let elem = fun elt =>
  lists.any (fun x => x == elt) in

let subList = fun l1 l2 =>
  elem (lists.head l1) l2
  && subList (list.tail l1) l2
\end{lstlisting}
\begin{lstlisting}[language=nickel,title={Optimized program}]
let subList = fun l1 l2 =>
  lists.any (fun x => x == (lists.head l1)) l2
  && subList (list.tail l1) l2
\end{lstlisting}
  \end{center}
\caption{Inlining}
\label{fig:optimizations-inlining-ex}
\end{figure}

While inlining expands an expression by substituting a definition for its value,
an opposite transformation is sometimes beneficial, for example when a composite
expression is repeated several times. This causes the same expression to be
wastefully recomputed at each occurrence. Common subexpression elimination (CSE)
consists in introducing a new identifier for this expression and using the
identifier in place of the original occurrences
(Figure \ref{fig:optimizations-cse-ex}), computing the result once and for all.

\begin{figure}[h]
  \begin{center}
\begin{lstlisting}[language=nickel,title={Source program}]
let elemAtOrLast = fun index list =>
  if index > lists.length list - 1 then
    lists.elemAt (lists.length list - 1) list
  else
    lists.elemAt index list
\end{lstlisting}
\begin{lstlisting}[language=nickel,title={Optimized program}]
let elemAtOrLast = fun index list =>
  let l = lists.length list - 1 in
  if index > l then
    lists.elemAt l list
  else
    lists.elemAt index list
\end{lstlisting}
  \end{center}
\caption{Common subexpression elimination}
\label{fig:optimizations-cse-ex}
\end{figure}

Beyond CSE, optimizations such as loop-invariant code motion or
let-floating~\ref{} apply the same principle of extracting an invariant
expression to avoid recomputation respectively across loop iterations and
function calls. For example, take the code of
Figure~\ref{fig:optimizations-let-floating-ex}.

\begin{figure}
  \begin{center}
\begin{lstlisting}[language=nickel,title={Source program}]
let f = fun x => g y (x + 1)
\end{lstlisting}
\begin{lstlisting}[language=nickel,title={Optimized program}]
let g' = g y
let f' = fun x => g' (x + 1)
\end{lstlisting}
  \end{center}
\caption{Let-floating}
\label{fig:optimizations-let-floating-ex}
\end{figure}

The partial application \lstinline+g y+ is recomputed each time \lstinline+f+ is
called. This may be costly, in particular in the presence of contracts: if
\lstinline+y+ is a list for example, and \lstinline+g+ applies a list contract
on it, the cost could be linear in the size of \lstinline+y+. A sensible thing
to do is to factor \lstinline+g y+ out of \lstinline+f+ as in
Figure~\ref{fig:optimizations-let-floating-ex}, which is something a
let-floating transformation could indeed do (given \lstinline+g+ is pure, as
explained below).

The soundness of all these optimizations is tied to the invariance of the
semantics of programs with respect to specific substitutions. Inlining requires
that one can replace the application of a function by its body, which is
basically $\beta$-reduction: as long as the arguments are evaluated following
the language's strategy, it is a valid transformation. On the other hand,
CSE-like transformations require an expansion property of the form

$$M[N] \simeq let~x~=~N~in~M[x]$$

This equation clearly fails in a call-by-value language with side effects:

\begin{lstlisting}[language=Nickel]
let f x = print "hi";(x+1) in (f 1,f 1)
(*@ $\not \simeq$ @*) let f x = print "hi";(x+1) in
(*@ \phantom{$\not \simeq$} @*) let y = f 1 in (y,y)
\end{lstlisting}

In the example above, the first term prints \nickel{"hi"} two times while the
second term prints it only once. However, this property does hold for
\emph{pure} terms, that are terms without side-effects. In a pure language, or a
language with effect tracking like Haskell\todo{Cite a different language, or
remove citation}, pure terms can be safely identified for applying
optimizations.

Strikingly, we will see in section~\ref{} that the introduction of unions and
intersections can make this equivalence \todo{name it} unsound \emph{even in a
pure setting}, making the optimization of
Figure~\ref{fig:optimizations-let-floating-ex} unsound in general.

\unsure{Should we tease already here that unions and intersections break this
even for pure terms?}

\subsubsection*{Lazyness}
\unsure{Should we talk about lazyness?
this would be a good place to introduce it}

\newpage

\section{Union \& intersection}
\info{What are they, what are they used for.}

\subsection*{Unions}
Unions are fundamental building blocks of programs. A union holds a value whose
shape is known to be in a fixed set of alternatives: for example, a value that
can be either a boolean or a string. They correspond logically to disjunctions.
Used in combination with products, they can represent arbitrary tree-like
structures, allowing to model a large class of data
(\ref{fig:union-adt-ex}).


\change{These next few labels need to be inside figures,
they are not working}
\label{fig:union-adt-ex}
\begin{lstlisting}[title={Tree-like data structure}]
type BinaryTree =
Integer | (BinaryTree, BinaryTree)
\end{lstlisting}


In addition to internal program representations, unions are also useful to model
external data:
\begin{itemize}
    \item To represent nullable values(\ref{fig:union-nullable-ex}).
    \item For plain enumerations (\ref{fig:union-enum-ex}).
    \item To support different representations of the same data
        (\ref{fig:union-alt-ex}).
\end{itemize}
All of which are common in configurations.

\label{fig:union-nullable-ex}
\begin{lstlisting}[title={Nullable values}]
optionalValue : String | Null
\end{lstlisting}

\label{fig:union-enum-ex}
\begin{lstlisting}[title={Plain enumeration}]
protocol : `Http | `Ftp | `Sftp
\end{lstlisting}

\label{fig:union-alt-ex}
\begin{lstlisting}[title={Alternative representations}]
person : {name : String, age : Int}
  | {name : String, dateOfBirth : String}
\end{lstlisting}

In statically typed languages, and more specifically in functional programming
languages, sums (also called tagged unions, or discriminated unions) are usually
preferred, as constructors for algebraic datatypes. [references]. Sums are unions' cousin where the alternatives are
syntactically separated by an associated discriminating label. Well-known
programming languages with first-class tagged unions include Haskell, OCaml,
Rust, Scala, PureScript, Elm, Swift and many more.

On the other hand, in dynamically typed languages, it is customary to directly
store different kinds of data in one variable without further ceremony. The
runtime system needs to attach type tags to values anyway, which can usually be
observed using functions like JavaScript's \verb+typeof+. Using such bare
unions\resolved{If you can observe with typeof, then it's pretty much a tagged
union, though the tag is implicit} has already been an idiom in dynamically
typed language for a long time. Being able to type and handle these unions is an
important aspect of the migration to a gradual type system. Indeed, a lot of
\unsure{This is the first time discussing gradual type systems, maybe it should
be introduced earlier, in particular how they relate to contracts 
(I think is mentioned on the appendix)}
gradually typed languages either existed originally as dynamically typed
language or aim at backward compatibility with a dynamically typed base
language. This is why despite the presence of a static type system, gradually
typed languages tend to favor bare unions types which is the typed counterpart
of this practice (as e.g in TypeScript \cite{TypeScriptUnions}).

In the case of Nickel, there is no pre-existing untyped language that we have to
support, so we are freer of our choice. However, Nickel is a configuration
language, whose contract and type system is intended to encode data schemas encountered in the wild. Since these schemas do use
untagged unions, there is a strong incentive to support untagged unions at the
type level, or at least, at the contract level.

\subsection*{Intersections}

Intersections are less prevalent. An intersection is a value that belongs to
two types at the same time. A simple example is a number literal \nickel{1}
which could be accepted as both an \nickel{Int} and a \nickel{Float} without
having to perform conversions (Figure~\ref{fig:intersection-subytping-ex}). In
this example, the intersection \nickel{Int & Float} can be used as a subtype of
both \nickel{Int} and \nickel{Float}.

\label{fig:intersection-subytping-ex}
\begin{lstlisting}[title={Value subtyping}]
1 : Int & Float
\end{lstlisting}

Intersections of higher-order types corresponding to overloading
(Figure~\ref{fig:intersection-overloading-ex}).

\label{fig:intersection-overloading-ex}
\begin{lstlisting}[title={Overloading}]
incr: (Int -> Int) & (Float -> Float) =
  fun x => x + 1
\end{lstlisting}

They can also be used to combine constraints (such as traits or interfaces), as
done in Scala 3 (Figure~\ref{fig:intersection-mixins-ex}).

\label{fig:intersection-mixins-ex}
\begin{lstlisting}[title={Structural mixins}]
Comparator: {cmp: T -> T -> Bool}
Shower: {show: T -> String}

showSmaller : (dict: Comparator & Shower) -> T -> T -> String
\end{lstlisting}

Finally, intersection naturally show up as the dual of unions. Thus, in a system
featuring unions, a union in a negative positions is a form of intersection
(Figure~\ref{fig:intersection-dual}).

\label{fig:intersection-dual}
\begin{lstlisting}[title={Intersection as the dual of unions}]
f : (Num | String) -> Bool
(*@ $\simeq$ @*) f : (Num -> Bool) & (String -> Bool)
\end{lstlisting}

\todo{Either here or the next section, show examples of unintuitive behaviour
of U and I. For example, I is not jos composition of contracts.
And union has to be treated as any one of the options}
% Things to settle: how to motivate them, how to compare them to ADTs. Purely
% dynamic approach, or typed approach too?

\newpage

\section{Issues coming from the semantics}
\info{What goes wrong or difficult when trying to implement union and
  intersection literally}

Type systems featuring intersection types are hard in a fundamental way,
from their discovery, they've been studied as a way to characterize
lambda-calculus terms normalizability; which means that typechecking a
term on a system with intersection types is undecidable,
similar to how the halting problem is undecidable\todo{reference}.

Usually, when thinking about programming languages, there's an intuition
that static checks (like typechecking) are more difficult and less powerful
than dynamic checks (like testing).
For instance, it is very easy to check if,
in a given execution of a program, a variable holds an integer value; while it
may become more challenging to check if a variable will always hold an integer value,
no matter the particular execution.

As we intend to show on this paper, unions and intersections are not only
statically hard to check, but they are also dynamically hard;
\emph{they are fundamentally hard}.
In this section, we present intuitively the main challenges that arise
when attempting to develop a runtime check system for unions and intersections.

\subsubsection*{Union contracts are hard to break}

When programming with dynamic checks it is assumed that lack of
failure does not mean that the program works perfectly.
For instance, imagine having a function wrapped in a contract
that checks that it maps positive numbers to positive numbers, like
function \nickel{f} on Figure \ref{fig:pos-to-pos}.

Clearly, there is an error on this program, but if the first call
to \nickel{f} that gets executed is \nickel{f 10}, the error will not
be found at that point, and the first operand of the addition would be
evaluated to \nickel{3}.
If, later on, \nickel{f 5} gets executed, the contract would get violated,
since \nickel{f} gets called with a positive value, but returns a negative
one (\nickel{-2}).

\begin{figure}[h]
\begin{lstlisting}[language=nickel]
let f = (fun x => x - 7) @ Positive -> Positive in
(f 10) + (f 5) 
\end{lstlisting}
\caption{Wrong positive to positive function}
\label{fig:pos-to-pos}
\end{figure}

Even if the program performed two different applications of \nickel{f}
before finding the contract violation, the first one is not actually
needed, and executing only \nickel{f 5} is enough to find the problem.

Having a witness of size 1 for every contract failure is a nice to have property
since it means, for example, that every time that a contract is violated,
the stack trace is enough to understand how it was violated, and to potentially
come up with a minimal example that breaks the contract.
However, when unions get introduced, it is lost.
Consider now the program presented in Figure \ref{fig:wrong-union-function},
this program (and in particular, \nickel{f}'s contract) still presents a problem,
since \nickel{fun x => x - 7} does not map positive numbers to positive numbers,
nor does it map positive numbers to non-positive numbers, so it won't be a valid
value for the union of those contracts.


\begin{figure}[h]
\begin{lstlisting}[language=nickel]
let f = (fun x => x - 7) @ (Positive -> Positive) | (Positive -> NonPositive) in
(f 10) + (f 5) 
\end{lstlisting}
\caption{Still wrong positive to positive or positive to non-positive function}
\label{fig:wrong-union-function}
\end{figure}

Still, the program presents enough function calls to find that issue
and violate the contract,
since \nickel{f 10} is a witness of \nickel{f} failing \nickel{Positive -> NonPositive},
and \nickel{f 5} is a witness of \nickel{f} failing \nickel{Positive -> Positive}.
But, what makes this case completely different from the one presented before, is that
both function calls are needed, and, what is more, there is not a way to
use the wrapped value \nickel{f} only once that would find a violation to its contract;
at least two are needed.

This seemingly simple consequence generates major issues when considered in real
life languages.
For example, remember the different mechanisms for code manipulation presented
in Section \ref{sec:optimizations}, inlining and commmon subexpression elimination,
both of these rely on the fact that substituting a pure function call by its body,
or joining equal pure expressions and sharing their result, is a semantically valid
transformation.
However, consider the examples presented at Figure \ref{fig:optimized-programs},
the first one is the inlined version of the second one, while the second
one has something like a CSE applied to it.

\begin{figure}[h]
\begin{lstlisting}[language=nickel, title=Inlined]
let f = (fun x y => if y then x else "False") @ Number -> (Boolean -> Number) | (Boolean -> String) in
(f 1 true, f 1 false)
\end{lstlisting}
\begin{lstlisting}[language=nickel, title=Common Subexpression Eliminated]
let f = (fun x y => if y then x else "False") @ Number -> (Boolean -> Number) | (Boolean -> String) in
let g x = f 1 x in
(g true, g false)
\end{lstlisting}
\caption{Equivalent programs, with inlining or CSE applied}
\label{fig:optimized-programs}
\end{figure}
\todo{Improve example}

Unexpectedly, this two programs behave differently, the first one will return
\nickel{(1, "False")}, without failing.
While the second one will fail, since the single application of \nickel{f 1},
applied to both \nickel{true} and \nickel{false} will be able to find the error.

Summing it up, wrapping a value with a union of higer order contracts means that
that value now has a state, that may get modified silently at every use site.
Both reviewed papers suffer from this defect since it is the semantics
of unions that interact badly with code manipulation operations,
not a particular implementation.
This addition of global state complicates both the work of the language
implementor, as well as the language user, since it adds non trivial
side effects to your code.
\unsure{Maybe mention CSE is actualy commonly done by hand as well}

\subsubsection*{[WIP] Flat contracts are too permissive}

Stop for a second and think how you could implement an intersection
contract. The first thought you may have is "just apply
both contracts", as shown on the Implementation of Figure \ref{fig:inter-contracts}.
Of course, if it were that simple we wouldn't be writing this paper,
this idea starts to show problems when you first combine it with
higher order contracts, since it would mean not complying with
at least one of these contracts negatively would mean breaking the
whole thing, as shown on the second example on the same Figure.

\begin{figure}[h]
\begin{lstlisting}[language=nickel, title=Implementation]
x @ A & B
(*@ $\simeq$ @*)
(x @ A) @ B
\end{lstlisting}
\begin{lstlisting}[language=nickel, title=Problem]
let g = (fun x => x) @ (Number -> Number) & (String -> String) in
g 1
\end{lstlisting}
\caption{Implementations of intersection contracts: first iteration}
\label{fig:inter-contracts}
\end{figure}

Still, not all hope is lost, we could pass some extra shared state
to each one of the sub contracts and let them only fail if the other
side of the contract has failed as well, as shown in Figure
\ref{fig:inter-contracts-2}, where the sared state is
represented by a label \nickel{l}.
This idea is still quite reasonable, for instance, when executing
\nickel{g 1}, from Figure \ref{fig:inter-contracts}, the first contract
that gets checked is the argument of \nickel{String -> String} over the function parameter, so
something like \nickel{1 @ String}.
Of course, this fails, but before raising a violation, this contract could
check if the other side of the intersection (\nickel{Number -> Number}) has
also tried to raise negative blame (on the parameter \nickel{1}), since
it hasn't, the contract would return normally, and would mark itself as failed.

\begin{figure}[h]
\begin{lstlisting}[language=nickel, title=Implementation]
x @ A & B
(*@ $\simeq$ @*)
(x @ A[l]) @ B[l]
\end{lstlisting}
\caption{Implementations of intersection contracts: second iteration}
\label{fig:inter-contracts-2}
\end{figure}

This idea, which is an oversimplification of the ideas
presented by Williams, Morris, and Wadler, seems to be spot on.
However, it presents a major problem, it's not compatible with flat contracts,
presented in Section \ref{sec:flat-contracts}.
Consider the example presented on Figure \ref{fig:inter-flat-contracts},
if the implementation presented previously were to be used, we may end up with
something like this \nickel{((fun x => x) @ (String -> String)[l]) @ C[l]}, where
\nickel{l} represents the shared state.
At this point, the intuitive thing to do is to check contract \nickel{C}, by checking
\nickel{((fun x => x) @ (String -> String)[l]) 0 == 0}.
This will fail, since applying \nickel{0} to a \nickel{String -> String} contract
fails negatively.

\begin{figure}[h]
\begin{lstlisting}[language=nickel]
let C = fun f => (f 0) == 0 in
let g = (fun x => x) @ (String -> String) & C
\end{lstlisting}
\caption{Intersection and flat contracts}
\label{fig:inter-flat-contracts}
\end{figure}

But this should not have happened, since in the program of Figure \ref{fig:inter-flat-contracts}
\nickel{g} is never applied to \nickel{0}, in fact, it is never applied.
Checking if \nickel{C} has already failed, and marking \nickel{String -> String} as
failed negatively on \nickel{l} is not enough, the contract \nickel{String -> String}
should not have been used inside the evaluation of \nickel{C}, which
adds an extra layer of complexity, contracts should be guarded, and execution
depends on the shape of the surrounding evaluation context.

\newpage

\section{Issues in concrete restrictions from the literature}
\info{The various papers and the tradeoffs they make}

The papers that previously researched union and intersection
as dynamic checks have, in our eyes, different shortcomings
that makes them suboptimal solutions.

\subsection{Keil and Thiemann}
\label{sec:keil-thiemann}
\unsure{Subsection title?}
Starting with the Keil and Thiemann work \cite{KeilThiemannUnionIntersection},
they noticed that, since union contracts should share information between
different application contexts, while intersection contracts should behave
independently between each application context, the system becomes much
simpler if every contract gets transformed into a union of intersections
of other contracts, by using the usual De Morgan laws,
and then eagerly opening up unions, but postponing intersection
contracts as much as possible.
\todo{This assumes the difficulty has been previously explained}

For instance, if the value \nickel{V} is wrapped with contract
\nickel{A & (B | C)}, then, before the contract is checked,
it will get opened up to something equivalent to
\nickel{(A & B) | (A & C)}.

This straightforward solutions makes composing the Keil and Thiemann
system with other existing systems more difficult; in the words
of Williams, Morris, and Wadler (see Section~\ref{sec:will-morr-wadl}):
``(\ldots) the monitoring semantics for contracts of intersection and union types given by Keil
and Thiemann are not uniform. (\ldots) If uniformity helps composition, then
special cases can hinder composition.''\cite{RootCauseOfBlame}
\info{(Yann) Does
it also make it less efficient? If $(A \& B)$ fails because of $B$, their
development causes to recheck the contract $A$ that has been duplicated?}
And it also could harm performance, or at least complicate an implementation,
since some sharing mechanism should be used to avoid checking the same contract,
over the same value, multiple times.\unsure{Arnaud says: this whole
  section is too abstract give concrete code and explain what goes
  wrong. It's fine to give code in Nickel syntax}

Another difficult issue Keil and Thiemann were able to solve, is the
fact that flat contracts and intersection do not usually play nice
together.
They provide two different sets of rules for their calculus,
the first one is non deterministic, whose purpose is to provide
semantics that are simple to understand.
The second one is a deterministic set of rules, that are intended
to be simple to implement.
This second set needs to perform operations
on the evaluation context, to decide whether or not to evaluate
a given contract.
\unsure{Worth showing how the non deterministic version works? (See section B.1.1)}

We see two problems here, a minor one is the fact that the two
different semantics defined for the calculus are really far apart
one from the other, both on the strategies used for contract resolution,
as well as on the complexity levels.
\unsure{(Yann) Is it really a
    problem per se? It is common thing to do (a nice declarative system
    for proofs and an algorithmic one for implementation, proved equivalent).
    Maybe the point is that the algorithmic system is way more complex than the
declarative one
(Teo) reworded}.
A second, major one,
is the dependency added over the evaluation context, since, again,
it makes composing this, with other features desired on programming languages,
difficult, or even impossible
\unsure{(Yann) I think we need to substantiate this claim. If this
context dependency explains why CSE is invalid, maybe have a little example?
(Teo) I removed it, I don't think there's an example, mainly since
this context dependency is not part of the language. I do think it complicates
implementation of said feature}
\todo{(Yann) TODO: see if there's a low hanging complexity
    argument we can make (like it's quadratic in the number of open contracts or
whatever)}.

\subsection{Williams, Morris, and Wadler}
\label{sec:will-morr-wadl}
\unsure{Subsection title?}

Williams, Morris, and Wadler, take a different approach on flat contracts,
they are completely forbidden in their system, explicitly to not encounter
the previously mentioned issue.
This restriction goes against what is usually expected from dynamic checks:
validation of data, and not just inclusion on types, is an expected
feature of contracts.

On top of that, a fundamental problem arises on said paper,
symptom of the difficulty of establishing good enough, and yet
simple semantics for union and intersection dynamic checks.
One of their main results, is the fact that their calculus obeys
all of their expected monitoring properties for higher ordered, union
and, intersection contracts.
However, one of these properties states the following:

$$ K \in \llbracket A \cap B \rrbracket^-~if~K \in \llbracket A \rrbracket^- \lor K \in \llbracket B \rrbracket^- $$

Which means that a continuation ($K$) complies negatively with an
intersection of $A$ and $B$ if it complies negatively with at least one of these.

Altough this is true, it's a weaker statement of what we'd like to have,
since it rules out continuations like $(\lambda f.~((f~3)~+~4,~not~(f ~ true)))$,
where $(\square 3)~+~4 \in \llbracket Number \rightarrow Number \rrbracket^-$
but $(\square 3)~+~4 \not \in \llbracket Boolean \rightarrow Boolean \rrbracket^-$,
and dually for $not~(f ~ true)$.
Which would mean, according to their semantics, that
the continuation presented above would not
be in $\llbracket Number \rightarrow Number \cap Boolean \rightarrow Boolean \rrbracket^-$,
even though our intuition (and the
introduction to the paper) indicates that it should.
\todo{Simplify example}

% since it rules out continuations like $(\lambda f.~(K_1[f],~K_2[f])) \square$,
% where $K_1 \in \llbracket A \rrbracket^-$ but $K_1 \not \in \llbracket B \rrbracket^-$,
% and dually for $K_2$, which would mean the continuation presented above would not
% be in $\llbracket A \cap B \rrbracket^-$, even though our intuition (and the
% introduction to the paper) indicates that it should.

While this mistake does not invalidate their work (the system they present
handles these cases correctly), it shows that existing research is still not on
good shape to handle the growing usecases being developed for
union and intersection around dynamic languages.

% $$ K ::= Id | K \bullet \square N | K \bullet V \square | K \bullet \square @^pA  $$

% Keil and Thiemann did noticed this problem, and their contract satisfaction
% rules only apply to elimination context, a particular kind of context
% where the hole is either being applied to an argument
% ($\square N$) or is used as an argument for a primitive
% operation ($O(\overrightarrow{\text{V}}, \square, \overrightarrow{\text{N}})$).



% K ::= Id | K ◦ □ N | K ◦ V □ | K ◦ □ @pA
\unsure{Is it worth mentioning how continuations are formed?
Should we go so techincal?}
\todo{(Yann) Maybe adding a concrete example of this (it can be the same term but with
    concrete types like Number and String and simple contexts like application
    to a dumb argument) is sufficient to make the point}

\subsection{Racket}
\label{sec:racket}

\unsure{Todo}


\newpage

\section{Related work}
\info{Including a mention of statically typed systems with union}
\info{Include something about Racket's and and or contracts}

\newpage

\section{Conclusion}

\bibliography{nickel}

\appendix

\todo{Remove parts that are already on the main part}

\section{Background on union types}

\subsection{Nickel design space}

{\color{red}Goals of the section

Introduce Nickel, explain the choice of gradual
typing and main design orientations, such as practice-oriented, lightweight,
etc.\vspace{0.5cm}}

Build systems, package managers, operating systems, cloud infrastructure,
continuous integration and web services\resolved{Also continuous integration}
are examples of modern complex software systems that require an extensive and
non-trivial configuration in order to make them adapted to each different
use-case. To manage the growing complexity that is then offloaded to
configurations, the declarative approach has become more and more popular,
illustrated for example by the infrastructure-as-code paradigm. This leads
configuration to play an important role in critical aspects of software
engineering, including security, availability, and maintainability.

However, static text-based configuration alone is falling short of
expressiveness, and is seldom sufficient. Data cannot be transformed, combined
nor shared, resulting at best in boilerplate and duplication of information, or
at worst, in data being invalid or inconsistent. Data validation is not
supported either, and must be handed over to yet another tool down the
configuration pipeline, if ever done. Correctly configuring a modern system is
hard and failures may have substantial negative consequences.

This is why the authors have been developing the Nickel
language\cite{NickelRepo}. Nickel is a configuration programming language,
meaning that a Nickel program evaluates to a value that must be
straightforwardly serializable to JSON, YAML or alike. Nickel aims at being
lightweight and easy to write while still providing features for writing correct
code and generating valid data. The focus is first and foremost practical:
Nickel originated as a rethink of the language of the Nix package manager, used
to describe one of the biggest sets of software packages\cite{repology}.  Nickel
targets build systems and cloud deployment tools (Kubernetes, Terraform or
NixOps) as well. While we are keen on incorporating existing or doing new
research when it solves a well-defined problem, the general design goals are
always guided by the practical industrial use cases.

\subsection{Typing}

Our aim is that writing simple Nickel feels as easy as writing JSON or YAML.
Unlike JSON, though, Nickel anticipates large configurations by being both
programmable and typed. In the configuration setting, there is a singular
dilemma with respect to typing: since a program is run on fixed inputs and is
expected to terminate, any relevant type error will show up at evaluation.  Why
bother with the complexity of a static type system? On the other hand, more and
more software systems offload complexity to configurations, in particular with
the infrastructure as code paradigm prevailing today in cloud deployment. When
the complexity of a codebase grows, static types become attractive again.  For
reusable code — that is, library functions —, static types are specifically
adapted, and bring all the usual benefits of early error detection, code
robustness, better code structure, documentation, and so on.

This dilemma is naturally solved by gradual typing\cite{Siek06gradualtyping}
which mixes both static and dynamic typing. Unlike gradually-typed languages
like TypeScript\unsure{Todo: explain that typescript uses unions as
  static typing whereas we are speaking of dynamic contracts}, we do not seek to build a complex type system that tries hard
to accept most of the idioms naturally arising in dynamically typed code.
Rather, we choose to provide a reasonably expressive type system with good
inference properties, which makes typing functions operating on generic data easy,
but may require using untyped code for more exotic expressions.  In any case,
this is not a surrender, as Nickel provides a complementary mechanism for more
advanced data validation: contracts.

\subsection{Contracts}
{\color{red}Goals of the section

A primer on contracts, which in the end are the problematic bit when confronted
with unions.\vspace{0.5cm}}

Enforcing pre- and post-conditions at runtime is a widely established practice.
In their foundational paper\cite{FindlerFelleisenHOContracts}, Findler and
Felleisen introduce a principled approach to run-time assertion checking that
nicely supports higher-order functions and introduces the notion of blame, which
is crucial to good error reporting. It became apparent later that their
contracts are closely related to the type casts introduced by gradual typing,
modulo blame: both \cite{FindlerMultiLang} and \cite{FelleisenInterLang} see the
value of contracts as a safe interface between typed and untyped code. In
\cite{WellTypedBlamed}, the authors precisely introduce a system integrating
gradual typing with contracts \textit{à la Findler \& Felleisen}.\unsure{The
historical bit should probably be moved to related works eventually} Nickel
adopts a similar type system, with both statically typed terms, dynamically
typed terms, and first-class contracts. In "first-class contracts", we include
the ability to write user-defined contracts. Because built-in types are too
simplistic for the purpose of data validation, the ability to write a small
validation function for, say, a port number, to have it immediately available as
a contract that can be composed with other type constructors\resolved{Needs code
to illustrate.}, is in our opinion one of the main appeal of Nickel's contracts
system. An example is given in Figure~\ref{fig:contract-composition-appendix}. The first
line defines a port contract from a predicate and use it in the definition of a
larger \lstinline+Configuration+ record contract. The \lstinline+Configuration+
contract checks that a value is a record with the following shape: it must have
a \lstinline+port+ field obeying the \lstinline+Port+ contract, and a
\lstinline+host+ field obeying a \lstinline+Host+ contract. It plays the role of
a \emph{schema}, used to validate the final data. The \lstinline+|+ operator
entails that a value respect a contract, akin to \lstinline+:+ for types. We
assume that a \lstinline+Host+ contract has been defined in a similar way.

\begin{figure}
  \begin{center}
\begin{lstlisting}[language=nickel]
let Port = contracts.fromPred (fun value =>
  builtins.isNum value && value % 1 == 0
  && value >= 0 && value <= 65353) in
let Configuration = {
  host | #Host,
  port | #Port,
} in
{host = "nickel-lang.org", port = 80}
| #Configuration
\end{lstlisting}
  \end{center}
\caption{Contract composition}
\label{fig:contract-composition-appendix}
\medskip
\small
Defines a port contract from a predicate and use it in the definition of a
\lstinline+Configuration+ record contract. \lstinline+Configuration+ plays the
role of a \emph{schema} for the final data. The \lstinline+|+ operator applies a
contract to a value, similar to `:` for static types. We assume that a
\lstinline+Host+ contract has also been defined.
\end{figure}

In this hybrid system, each type constructor of the static type system - arrows,
records, foralls - must have a contract counterpart that checks at run-time that
a value is a member of the corresponding type. Contracts for various
extensions of the simply typed $\lambda$-calculus have been investigated
\cite{BlameForAll, KeilThiemannUnionIntersection, RootCauseOfBlame,
DependentContracts, GradualTypingClasses}, with solutions of varying complexity.
Amongst all the common extensions to the polymorphic lambda calculus, one of the
most useful and widespread turns out to be one of the most challenging:
\emph{union types}.

\subsection{Union types}
{\color{red}Goals of the section

Introduce union in general, and most usual form in statically typed languages, tagged unions\vspace{0.5cm}}

\todo{Specify: union types? more general concept?}{Unions} are fundamental and ubiquitous building blocks of program data. A union
only holds one value whose shape is only known to be in a fixed set of
alternatives: for example, either a boolean or a string. When used in combination
with products, they can represent arbitrary tree-like structures, allowing to
model a large class of data.

Unions are also useful for configurations:
\begin{itemize}
    \item To represent nullable values in JSON.
    \item For plain enumerations. For example, the enabled protocols of a data server
        would be either \lstinline+HTTP+, \lstinline+FTP+ or \lstinline+SFTP+.
    \item To support different representations of the same data. A file path could be
        accepted either as one string of \lstinline+"/"+-separated strings, or
        as a list of strings.
\end{itemize}

In statically typed languages, and more specifically in functional programming
languages, unions are usually implemented as algebraic data
types\unsure{This is a very misleading statement at best. Sums are
  "tagged union" sure, but that refers to an encoding a sums with
  unions, certainly not an encoding of unions with sums!} [references],
also called \emph{tagged unions}. Well-known programming languages with
first-class tagged unions include Haskell, OCaml, Rust, Scala, PureScript, Elm,
Swift and many more.

\subsection{Tagged versus untagged}
{\color{red}Goals of the section

Interpreted languages already need to tag their values, hence bare union are more natural.
It also corresponds better to prior usage in untyped code\vspace{0.5cm}}

In the statically typed setting, the actual representation of a tagged union is
an implementation detail: from the point of view of the programmer, a tagged
union is a first-class data type associated with constructors and destructors
with well-defined semantics. In contrast, in a gradually typed language like
Nickel, the question of the representation of tagged unions as untyped values
arises.

Unfortunately, tagged unions do not have a serializable canonical counterpart as
untyped data. One usual representation is a record with a tag and a value whose
shape depends on the tag (see Figure \ref{fig:union-encoding}).

\begin{figure}
  \begin{center}
\begin{lstlisting}[language=nickel]
{tag: <Num, Str>, value: Dyn}
\end{lstlisting}
  \end{center}
\caption{Encoding of the union type: integer or a string}
\label{fig:union-encoding}
\medskip
\small
In Nickel, \lstinline+<A, B, C>+ is the type of a C-like enumeration that can
take value \lstinline+A+, \lstinline+B+ or \lstinline+C+, while \lstinline+Dyn+
is the dynamic unitype.
\end{figure}

A better way to describe it would be as a dependent pair which first component
is a tag and second component is the data type corresponding to this tag:

\[
    \Sigma_{tag : <A, B, C>} \text{Alternative}(\text{tag})
\]

Nickel does not feature dependent types, which would open a whole new level of
complexity. Alas, choosing the non dependent representation as a default is not
fully satisfying.  First, it does not map precisely to a JSON value: in
consequence, a lot of illegal combinations are structurally valid as untyped
JSON values. But this is to be expected of any representation. More importantly,
this does not necessarily match with the configuration schemas one can find in
the wild, which favor simple, untagged union.

In dynamically typed languages, it is customary to directly store different kinds
of data in one variable without further ceremony. The runtime system needs to
attach type tags to values anyway, which can usually be observed using functions
like JavaScript's \verb+typeof+. Using so-called \emph{untagged
  unions}\unsure{If you can observe with typeof, then it's
  pretty much a tagged union, though the tag is implicit} has
already been an idiom in dynamically typed language for a long time
[ref/examples], and being able to type and handle these unions is an important
aspect of the migration to a gradual type system. Indeed, a lot of gradually
typed languages either existed originally as dynamically typed language or aim
at backward compatibility with a dynamically typed base language. This is why
they tend to favor supporting untagged union types, the typed counterpart of
this practice (as e.g in TypeScript \cite{TypeScriptUnions}).

In the case of Nickel, there is no pre-existing untyped language that we have to
support, so we are freer of our choice. However, Nickel is a configuration
language, whose contracts and types system is intended to be expressive enough
to encode data schemas encountered in the wild. Since these schemas do use
untagged unions, there is a strong incentive to support untagged unions at the
type level, or at least, at the contract level.

\subsection{Tagged as untagged}
{\color{red}Goals of the section

Untagged unions, modulo a slightly smarter typechecker, can encode usual type-safe
tagged unions. They let the user free of the representation though, and encode
more stuff, that's why they are appealing.\vspace{0.5cm}}

Untagged union, together with the other existing features of the type system,
are sufficient to implement the encoding illustrated in Figure
\ref{fig:union-encoding}. For example, let us represent an algebraic data type
(written in ML syntax) as an imaginary Nickel union type:

\begin{lstlisting}[language=caml]
type either = Left of float | Right of string
\end{lstlisting}

\begin{lstlisting}[language=nickel]
{ tag : <Left>, val : Num}
| { tag : <Right>, val : Str}
\end{lstlisting}

There is a twist, though. As it is, the static type system is unable to refine
the type of the value, making the following example rejected:

\begin{lstlisting}
switch x.tag {
  Left => x.val + 1,
  Right => strings.fromNum x.val
}
\end{lstlisting}

Flow typing, implemented in Racket\cite{FlowTypingRacket1, FlowTypingRacket2},
Groovy\cite{FlowTypingGroovy}, Whiley\cite{FlowTypingWhiley} or
TypeScript\cite{FlowTypingTypeScript}, is capable of refining appropriately the
type information in each branch. Thus equipped with untagged unions and a
typechecker smart enough, classic tagged unions can be simulated, but with the
additional benefit that the language is not imposing any representation.
Untagged unions empower the programmer to represent and handle a wider variety
of union idioms appearing in the wild.  Thus, adding union types to Nickel
sounds appealing.

\subsection{Union and intersections}
{\color{red}Goals of the section
Explain why having unions probably unavoidably leads one to have intersection.\vspace{0.5cm}}

While our original motivation is in supporting unions, the rest of the paper
explore complete systems featuring union and intersection types. One could
wonder if the difficulties could then be sidestepped by removing intesection out
of the equation. But

\begin{enumerate}
    \item Most of the encountered issues already appear with bare unions.
    \item Intersection being dual to unions, they are showing up in some way
        when unions are put in negative positions. For example, a contract for
        $(A \cup B) \to C$ in a system without intersections must be
        equivalent\unsure{``Must be'' is a rather strong statement, we
        will find a way to tone it down}
        to the contract $(A \to C) \cap (B \to C)$
\end{enumerate}

\unsure{Some bits missing so far: we are interested in higher-order
  contracts, Nickel is a higher-order \& pure language, contracts as
  functions. We've said that user-defined contracts are a thing, it
  probably matters so we may have to hammer it down.}

\unsure{Something that we must mention somewhere: if we want to check
  that something is a list of As lazily (to avoid making functions
  like hd be O(n), for instance), then we have a lot of the problems of
  higher-order contracts anyway.}

\subsection{Performance considerations}
\info{This a draft}
\todo{transition, location}

At first sight, performance should not be a critical issue given the use cases
of Nickel. But generating large configurations can already hit some limits in
current related languages [reference?]. A natural is parallelization:
Nickel being an almost pure functional programming language, evaluating separate
files or expressions in parallel is straightforward and effective. This
parallelization potential is thus a design requirement, and must be when the
addition of union types.

\section{Challenges}
\newcommand{\moral}[1]{\noindent $\hookrightarrow$ \textsc{#1}}

% Type systems featuring intersection types are hard in a fundamental way,
% from their discovery, they've been studied as a way to characterize
% lambda-calculus terms normalizability; which means that typechecking a
% term on a system with intersection types is undecidable,
% similar to how the halting problem is undecidable\todo{reference}.

% Usually, when thinking about programming languages, there's an intuition
% that static checks (like typechecking) are more difficult and less flexible
% than dynamic checks (like testing). For instance, it is very easy to check if,
% in a given execution of a program, a variable holds an integer value; while it
% may become more challenging to check if a variable will always hold an integer value,
% no matter the particular execution.

% As we intend to show on this paper, union and intersection types are not only
% statically hard to check, but they are also dynamically hard;
% \emph{they are fundamentally hard}.
% In this section, we present intuitively the main challenges that arise
% when attempting to develop a runtime check system for union and intersection types.

% \subsection{Known challenges}

% There are two works proposing solutions to dynamically checking union and intersection
% types (through contracts)\todo{reference}. 
% We briefly review the main challenges they consider, and some extra ones.
% At the same time, we outline the reasons that make these works,
% in our opinion, non satisfactory.

% \todo{add Castagna "Gradual types a new perspective"}

% \subsubsection*{Union contracts may need multiple evaluations}

% When using contracts, programmers assume that lack of blame does not mean
% lack of errors.
% For instance, imagine you have a function
% wrapped with a contract asserting that it sends positive values to positive values,
% ($let~f~=~(\lambda x.~x~-~7)@Pos \rightarrow Pos$), clearly, this contract
% does not hold.
% Still, applying $f~10$ returns $3$, complying with the contract.
% Contracts perform a dynamic check, similar to how testing works.

% However, a call like $f~5$, would return $-2$, raising an error since
% the contract has been violated.
% Even if only after evaluating both of this function calls
% the error was found, the first one is not really needed, and $f~5$ is enough to
% prove that $f$ doesn't comply with the given contract.
% \unsure{Arnaud: Instead
%   of speaking about it in the abstract, give me concrete programs
%   which are difficult.
%   Teo: Not sure I understand what do you mean by difficult programs,
%   my intention here is just to show that classic contracts need at most 1 witness}


% Having a witness of size 1 for every contract failure is a nice to have property,
% yet when unions get introduced, it is lost.
% Imagine wrapping that same function with a contract that states that it either
% sends positive values to positive values, or positive values to non positive values
% ($let~g~=~(\lambda x.~x~-~7)@(Pos \rightarrow Pos) \cup (Pos \rightarrow NonPos)$),
% clearly, this contract should fail, yet there is not one single application
% of $f$ that would make it blame, at least two are needed ($(f~10,~f~5)$).


% % \moral{Union contracts behave globally}

% This seemingly simple consequence breaks other major properties
% of functional languages.
% Consider the following programs:
% $let~g~=~(\lambda x.~x~-~7)@(Pos \rightarrow Pos) \cup (Pos \rightarrow NonPos)~in~(g~10,~g~3)$
% and
% $((\lambda x.~x~-~7)@(Pos \rightarrow Pos) \cup (Pos \rightarrow NonPos)~10,~(\lambda x.~x~-~7)@(Pos \rightarrow Pos) \cup (Pos \rightarrow NonPos)~3)$.
% Transforming the first one to the second one is what's usually considered as \emph{inlining},
% the other direction would be \emph{common subexpression elimination}.

% Both of these operations are heavily used on functional languages as optimizations,
% yet union contracts make their semantics to differ, the first case will violate the contract
% and fail, yet the second one will complete execution and return $(3, -4)$.

% Both reviewed papers suffer from this defect since it is the semantics
% of unions that interact badly with inlining/common subexpression elimination,
% not a particular implementation.

\subsubsection*{Intersection contracts must not share information}

\info{We know it's hard if you have solved the issue with union, then
  it's easy to have this example not type.}
\info{Is there a synthetic example that illustrates the difficulty
  more obviously. Maybe $f = \lambda x. if x == 0 then true else 4 @
  Any \& ((N -> N) | (N -> B)) in (f 0, f 1)$ or $f = \lambda x. if x == 0 then true else 4 @(N -> (N | B)) \& ((N -> N) | (N -> B)) in (f 0, f 1)$}
Intersection types need a dual property, that is, the contract
should be postponed as much as possible, and a value contracted with some intersection
should independently check every evaluation context 
\change{Same: what property?}.

Imagine having a function, with some unimportant implementation,
and wrapping it
with a contract stating that it behaves both as a $Bool$ to $Bool$ function,
and as a $Num$ to $Num$ function; that is, 
$let~f~=~...@(Bool \rightarrow Bool) \cap (Num \rightarrow Num)~in~(f~42,~f~"hello")$.
Now, the first application ($f~42$) invalidates the first contract, but complies with the second
one by applying a number; the second application ($f~"hello"$), behaves dually,
it invalidates the second contract,
while complying with the first one.

This example shouldn't fail, even if the whole program invalidates both choices of $f$, it does
so at two different points of the execution and it should not raise an exception.
Therefore, the elimination of an intersection contract must be local to each elimination
context, as pointed out by Keil and Thiemann \cite{KeilThiemannUnionIntersection}.

\moral{Intersection contracts behave locally}\\

\info{Maybe another way of saying it is each instance must be executed in a
separate environment}
% GOAL: State that intersection and union, without HO, is easier. And that is being implemented on nickel.

As one dives into the problems that may become present when dealing with union and intersection,
a valid question is how these could be implemented in the simplest of cases.
For instance, imagine having
union and intersection only present for lower ordered types 
($Bool$, $Num$, $String$, $(Num, Bool)$, etc.), then checking
if a particular value is a $String$ or a pair of $Num$ and $Bool$ is easy,
just check both and fail only if both checks fail.
Checking if a value complies with an intersection of two contracts, $A$ and $B$ is even
easier, just apply both contracts independently, one after the other.

Typed Racket implements this idea, but calls intersection and union contracts
\texttt{and} and \texttt{or}, respectively.

\moral{It is the combination of union, intersection, and function contracts that is not trivial}
\change{The moral is true, but maybe restate the problem are functions. Do we
    want to say that since Nickel is lazy, almost every value is morally a
function (thunk)?}

% \subsubsection*{Flat contracts are not idempotent}

% Following the simple case presented above, there's another kind of extension that could be
% applied to a contract system, and that is contracts written by the
% programmer to check arbitrary properties; these are usually called flat contracts and
% involve a predicate over the contracted value that returns a boolean, indicating whether
% the contract passed or it should raise an exception.

% For instance, one may build a contract that checks that the value in question is a
% function that, when applied to $1$, returns $1$, as something like
% $\lambda f.~f~1~==~1$; let us call it $C$.
% Then, wrapping a given value to check that it complies with this contract, but that at
% the same time is a valid $String \rightarrow String$ function may look something like this
% $let~f~=~...@C \cap (String \rightarrow String)$, a valid implementation for this function
% may be $\lambda x.~x$.
% However, depending on how this intersection is resolved, we may face the problem of applying first
% the $String$ to $String$ contract, and then $C$, in which case we would try to apply a function,
% wrapped with a $String \rightarrow String$ contract, to 1, raising blame on the context
% of the contract execution.\unsure{I think that the point here is that
%   $f@C \cap (String \rightarrow String)$ isn't the same as $f@(String
%   \rightarrow String) @C$}

% To solve this problem, each execution of a contract has to be aware of its execution context,
% or of its procedence.


% % TODO come up with an example not using HO contracts, we need to present Nickel's flat contracts,
% % that are much more permissive

% \moral{The addition of flat contracts involves evaluation context dependency.}

% Of course, one could ask himself whether applying these two contracts separately,
% could make any sense, but this bring either some kind of exponential explosion
% on the execution, or a need for a non deterministic execution
% strategy.
% The authors of this papers spent some time trying to solve this by using
% \texttt{call/cc}, an attempt that became too unpractical too rapidly.

% Keil and Thiemann propose a non deterministic strategy for their contract checking
% calculus; this gives a very simple and elegant solution, but, sadly, still
% unfeasible to implement\unsure{Strong statement, can we back this up?
%   I think we want to explain what Keil and Thiemann looks like}.

% The gist of their proposal, is to have a calculus that allows non deterministic
% executions, that are joined if they reach an equal term, so $M \parallel M$ gets
% rewritten as $M$.
% So, for instance, when evaluating $(M \parallel N) ~ T$, it gets
% reduced to $ M ~ T \parallel N ~ T$.
% And non determinism gets introduced when an intersection contract gets applied
% over a value, that is being applied to another value, $(V @ Q \cap R) ~ W$
% gets rewritten as $(V @ Q) ~ W \parallel (V @ R) ~ W$\footnote{This is overly
% simplified version of their system, only focused on how non determinism behaves.}.

% This approach is a literal reading of the logical semantics of intersection, where
% a context is free to chose any one of the intersected contracts.
% In this case, both are tried, and computation only fails negatively if both
% threads of execution do.

% They also propose a different approach, by checking the context at which
% each contract is opened, dropping
% it if the context indicates it is being applied inside the evaluation of another
% (related by an intersection) contract.
% This check implies traversing the context of execution.
% We consider that having execution dependent on the context
% of evaluation breaks locality of
% the language execution in undesired ways.\todo{Unclear. Is it a bad thing
%     "philosophically" (in general, because it breaks local reasoning?) If there
% are concrete unwiding examples, we should add some}.

% \unsure{The next paragraph is a draft that tries to answer the todo comment above}
% Breaking locality on the semantics of a programming language can unwind
% multiple undesired consequences, for instance, put yourself in the shoes
% of a compiler programmer, it does not matter whether your language is eager
% (C) or lazy (Haskell) you don't want to waste time evaluating
% the same exact thing multiple times.
% The dependency added in KT to the context of evaluation forces their system
% to treat a value, wrapped in a function contract (or an intersection of function
% contracts) as a value by itself. This means that the contract needs to be reopened
% at every evaluation context.

% WMW decided to disallow flat contracts completely,
% we believe this hurts badly a dynamically checked framework, since it goes against
% the common practice, in dynamic languages, of checking for properties
% that go beyond the expressiveness given by a type system.
% \todo{Maybe mention explicitly the word "validation": contracts allow for
%     principled, higher-order compatible, composable validation. For validation,
% you want to check things much more complex than just belonging to a type.}

% \subsubsection*{Inflexibility of union and intersection contracts}

% Solutions for union and intersection contracts do exist (KT, WMW) however,
% they do not provide the flexibility that one may be used to when using dynamic checks.
% Consider a Python programmer, she would not care to see something like this
% \texttt{assert(isinstance(3 if True else "hello", int)))}, while a Haskell programmer
% might perform a harakiri upon reading it.

% Without falling into the eternal discussion of static vs dynamic type checking,
% regular users of dynamically checked languages expect this flexibility.
% Sadly, the current work on union and intersection does not provide it.

% For instance, consider the following code
% % \texttt{(\\f.(\\xy.x) (f 1) (f true))((\\xy.x)@(N -> N -> N) /\\ (N -> B -> N) 1)}.
% $let~f~=~((\lambda~x~y.x)@(N -> N -> N) \cap (N -> B -> N))~1~in~(f~1,~f~True)$.
% If we use any of the KT or WMW frameworks, both our Haskell programmer and our
% Python programmer will change their career path towards photography, or something
% similar.
% Our Haskell programmer, will do so because this code compiles;
% while the Python programmer would do so because this code fails on execution.

% The reason this happens, is that even if you would expect
% $(N -> N -> N) \cap (N -> B -> N)$
% and $N -> (N -> N) \cap (B -> N)$ to behave equally, the point at which
% the decission of which path of the intersection to use is taken
% matters.
% When working with dynamic checks, it is expected that this kind of
% flexibility works.

% \todo{better phrasing and try to use the dual example}

% \moral{Are union and intersection contracts really what we want? Is there an alternative (simpler) option?}

% \subsection{Complexity of existing solutions}

% The Keil and Thiemann paper presents a complete system that handles union and intersection
% contracts in the presence of higher order contracts, however, we consider it to be overly
% complicated for a reasonable to implement language.
% In what follows, we state some of the reasons that we consider troublesome.

% \subsubsection{Non deterministic semantics}

% The semantics of the calculus are presented, primarly, on a non deterministic interpretation,
% which means that, even if their work provides a deterministic semantics,
% the expected way to understand
% the behavior is not the same as the expected way to implement it, cluttering the development
% of a language that could provide this solution.

% Keil and Thiemann had also developed a contract system for JavaScript, called TreatJS, that
% implements their ideas, but it does so with a different technique that the one presented
% on the paper.

% On our opinion, having a formalization that lies far away from the implementation
% hearts any future expansion work on the system, and it is not a satisfactory solution
% for what we intended to pursuit.


% \todo{Maybe mention call/cc}

% \subsubsection{Context opening}

% Keil and Thiemann noticed that flat contracts cannot be arbitrarily used, since
% violating a flat contract inside another contract would, unexpectedly, raise blame.
% Their solution, is to check the context at which each contract is opened, dropping
% it if the context indicates it is being opened inside the evaluation of another
% contract.

% This check, implies traversing the context of execution.
% We consider that having execution dependent on the context
% of evaluation breaks locality of
% the language execution in undesired ways.

% \subsubsection{Contract normalization}

% GOAL: The relation between union and intersection have to be modified, transforming them
% in a union of intersections.

% In the Keil and Thiemann paper, before the validation of contracts can begin, the
% contracts have to be normalized into unions of intersections of (other) contracts.
% This problem  is the starting point from which
% WMW deliver their work, and we share their opinion:

% "However, the monitoring semantics for contracts of intersection and union types given by Keil
% and Thiemann are not uniform. (...) If uniformity helps composition, then
% special cases can hinder composition.", [Ref]

% \subsection{Shortcomings in WMW}

% The second paper worth mentioning is the one by Williams, Morrison and Wadler [ref].
% This work, attempts to provide a simpler contract system with
% union and intersection types.
% However, we believe this work has some issues and shortcomings, which make it
% not that useful in practice; we briefly outline them below.

% % \subsubsection*{Lack of flat contracts}

% % In order to save themselves from dropping some contracts when checking a union
% % or intersection branch, WMW decided to eliminate completely flat contracts from
% % their system.
% % We believe this hurts badly a dynamically checked framework, since it goes against
% % the common practice, in dynamic languages, of checking for non trivial properties.

% \subsubsection*{Problem in semantics (monitoring)}
% Is the problem in the semantics worth mentioning? or is it attacking
% too much on the paper.\unsure{I don't know, show it to me first and we
% can decide!}


\section{Current (Real life) implementations}

\info{The goal of this section is to outline existing languages that implement
union and intersection, in some form, and shortly compare these
with the semantics tried to obtain on research (WMW, KT)}

Unions and intersection are starting to become more and more common
on non-academic programming languages, as stated on the introduction,
unions are a natural abstraction for dynamic languages, where the type
of an expression could have many different shapes, without necessarily
having a mark to distinguish between them at type level.
With this in mind, a valid question is how do these languages solve the
union/intersection conundrum? Or, even better, what is it that they
solve?

In the rest of this section we revisit some different approaches.

\subsection{Racket}

Racket is a language based on Scheme, mainly thought as a language to experiment
with different language desings and ideas. It provides a very complete contract
system\cite{RacketContracts}.

Among those contracts, they provide \racket{and/c} and \racket{or/c} contracts.
These contracts are lightweight versions of intersections and unions, they simply
check that every contract is valid, in the case of \racket{and/c},
and that at least one contract
is valid, in the case of \racket{or/c}.

So, for instance, the usual usecase of intersection to encode bounded overloading
does not work on Racket. Take a look at the next example, the function
\racket{overload} should accept a call like \racket{(overload 3)}, but it fails
since 3 is not a valid \racket{string?}.

\begin{figure}[h]
 
\begin{lstlisting}[language=racket]
(define/contract overload
(and/c (-> number? number?)
       (-> string? string?))
(lambda (x) x))
\end{lstlisting}
\caption{Non working function overloading on Racket.}

\end{figure}

Even worst, the \racket{or/c} contract doesn't allow to have multiple higher order
contracts that could (potentially) apply to a function, these have to be
differentiated by first order markers.
For instance, the code on Figure~\ref{code:racket:or/c:working}
would correctly allow only to call
\racket{united} with a \racket{number?},
and it can make this distinction since it knows that
\racket{(lambda (x) x)} can not be a function that takes 2 parameters.
However, the code on Figure~\ref{code:racket:or/c:non-working} would fail
before running, since the \racket{or/c} contract is unable to know
which of the two function contracts to use.
Ideally, this last example should just behave as wrapping
the function with the \racket{(-> even? even?)} contract.

\begin{figure}[h]
  
\begin{lstlisting}[language=racket]
(define/contract united
(or/c (-> number? number?)
      (-> string? string? string?))
(lambda (x) x))
\end{lstlisting}
\caption{Working \racket{or/c} example on Racket.}
\label{code:racket:or/c:working}

\end{figure}

\begin{figure}[h]

\begin{lstlisting}[language=lisp]
(define/contract united
(or/c (-> number? number?)
      (-> even? even?))
(lambda (x) x))
\end{lstlisting}
\caption{Non working \racket{or/c} example on Racket.}
\label{code:racket:or/c:non-working}

\end{figure}


\unsure{This section is overall a bit hard to follow. I
  think I've figured out what it means, but it needs more love. Also
  give an example which doesn't work and one that does for or/c.}

\subsubsection*{case->}

Racket does provide an alternative for overloading methods, called the
\racket{case->} contract, however, this overloading of methods only
works over contracts of functions with different amount of parameters.
It's a good alternative to allow limited overloading of functions.
As an example, consider the following non failing piece of code, and
try to understand how Racket can decide, at each application,
which function contract to use.

\begin{figure}[h]

\begin{lstlisting}[language=racket]
(define/contract overcase
  (case-> (-> string? string?)
          (-> number? number? number?)
          )
  (lambda (x [y 0]) (if (number? x)
                   (+ x y)
                   x)))

(overcase 1 2)

(overcase "hello")
\end{lstlisting}
\end{figure}

\todo{Coming back to the problems of composing U/I with other typing
features, Racket has made a clear choice and decided that higher order
is more important (a wise decision)}

\subsection{Python - MyPy}

\unsure{We ought to group all the static typing in one subsection}

\end{document}

% LocalWords:  Felleisen
