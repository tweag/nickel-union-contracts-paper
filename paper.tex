\documentclass{article}
\usepackage{mathtools,amssymb,amsmath,amsfonts,amsxtra,amsthm,mathrsfs,mathpartir}
\usepackage{listings}
\usepackage{xcolor}
\usepackage{array}
\usepackage[toc,page]{appendix}
\usepackage{graphicx}
\usepackage{varioref}
\usepackage{verbatim,comment}
\usepackage{longtable}
\usepackage{tikz}
\usepackage{url}

\usepackage{biblatex}
\addbibresource{nickel.bib}

\title{Contracts for unions and intersections are hard for Nickel}
\author{Teodoro Freund, Yann Hamdaoui, Arnaud Spiwack}

\begin{document}

\maketitle

\begin{abstract}
This paper reports an attempt to incorporate union and intersection types in
Nickel, a configuration programming language with a gradual type system and
first-class contracts. While the end result looks appealing, it turns out the
concrete impact on both the design and the implementation of the language is
unexpectedly high. We review the issues raised in particular by the
implementation of contracts for unions and intersections, and why we think the
cost ends up too high for the benefit. We conclude by suggesting different leads
to represent and type unions.
\end{abstract}

\section*{Introduction}
Intro

\section{Background on union types}

\subsection{Nickel design space}

\color{red}Goals of the section

Introduce Nickel, explain the choice of gradual
typing and main design orientations, such as practice-oriented, lightweight,
etc.\vspace{0.5cm}\color{black}

Build systems, package managers, operating systems, cloud infrastructures, and
web services are examples of modern complex software systems that require an
extensive and non-trivial configuration in order to make them adapted to each
different use-case. To manage the growing complexity that is then offloaded to
configurations, the declarative approach has become more and more popular,
illustrated for example by the infrastructure-as-code paradigm. This leads
configuration to play an important role in critical aspects of software
engineering, including security, availability, and maintainability.

However, static text-based configuration alone is falling short of expressivity,
and is seldom sufficient. Data cannot be transformed, combined nor shared,
resulting at best in boilerplate and duplication of information, or at worst, in
data being invalid or inconsistent. Data validation is not supported either, and must be handed over to yet another tool down the configuration pipeline, if ever
done. Correctly configuring a modern system is hard and failures may have
substantial negative consequences.

This is why the authors have been developing the Nickel
language\cite{NickelRepo}. Nickel is a configuration programming language,
meaning that a Nickel program evaluates to a value that must be
straightforwardly serializable to JSON, YAML or alike. Nickel aims at being
lightweight and easy to write while still providing features for writing correct
code and generating valid data. The focus is first and foremost practical:
Nickel originated as a rethink of the language of the Nix package manager, used
to describe one of the biggest sets of software packages\cite{repology}.  Nickel
targets build systems and cloud deployment tools (Kubernetes, Terraform or
NixOps) as well. While we are keen on incorporating existing or doing new
research when it solves a well-defined problem, the general design goals are
always guided by the practical industrial use cases.

\subsection{Typing}

Our aim is that writing simple Nickel feels as easy as writing JSON or YAML.
Unlike JSON, though, Nickel anticipates large configurations by being both
programmable and typed. In the configuration setting, there is a singular
dilemma with respect to typing: since a program is run on fixed inputs and is
expected to terminate, any relevant type error will show up at evaluation.  Why
bother with the complexity of a static type system? On the other hand, more and
more software systems offload complexity to configurations, in particular with
the infrastructure as code paradigm prevailing today in cloud deployment. When
the complexity of a codebase grows, static types become attractive again.  For
reusable code — that is, library functions —, static types are specifically
adapted, and bring all the usual benefits of early error detection, code
robustness, better code structure, documentation, and so on.

This dilemma is naturally solved by gradual typing\cite{Siek06gradualtyping}
which mixes both static and dynamic typing. Unlike gradually-typed languages
like TypeScript, we do not seek to build a complex type system that tries hard
to accept most of the idioms naturally arising in dynamically typed code.
Rather, we choose to provide a reasonably expressive type system with good
inference properties, which makes typing functions operating on generic data easy,
but may require using untyped code for more exotic expressions.  In any case,
this is not a surrender, as Nickel provides a complementary mechanism for more
advanced data validation: contracts.

\subsection{Contracts}
\color{red}Goals of the section

A primer on contracts, which in the end are the problematic bit when confronted
with unions.\vspace{0.5cm}

\color{black}

Enforcing pre- and post-conditions at runtime is a widely established practice.
In their foundational paper\cite{FindlerFelleisenHOContracts}, Findler and
Fellleisen introduces a principled approach to run-time assertion checking that
nicely support higher-order functions and introduce the notion of blame, which
is crucial to good error reporting. It became apparent later that their
contracts are closely related to the type casts introduced by gradual typing,
modulo blame: both \cite{FindlerMultiLang} and \cite{FelleisenInterLang} see the
value of contracts as a safe interface between typed and untyped code. In
\cite{WellTypedBlamed}, the authors precisely introduce a system integrating
gradual typing with contracts \textit{à la Findler \& Felleisen}. Nickel adopts
a similar type system, with both statically typed terms, dynamically typed
terms, and first-class contracts.  In this setting, each type constructor of the
static type system - arrows, records, foralls - must have a contract counterpart
that check at run-time that a value is indeed a member of the corresponding
type. Contracts for various extensions of the simply typed $\lambda$-calculus
have been investigated \cite{BlameForAll, KeilThiemannUnionIntersection,
RootCauseOfBlame, DependentContracts, GradualTypingClasses}, with solutions of
varying complexity.  Amongst all the common extensions to the polymorphic lambda
calculus, one of the most simple and ubiquitous turns out to be one of the most
involved: \emph{union types}.

\subsection{Unions types}
\color{red}Goals of the section

Introduce union in general, and most usual form in statically typed languages, tagged unions\vspace{0.5cm}\color{black}

Products and unions are fundamental and ubiquitous building blocks of program
data. A product is a compound data that hold several values of potentially
different shape packed together. Dually, a union only holds one value whose
shape is only known to be in a fixed set of alternatives. When nested together,
unions and intersections are able to represent arbitrary tree-like structures,
allowing to model a large class of data.

Unions are also useful for configurations, to represent nullable values in JSON,
plain enumerations, or to support several representations for a field (for
example, a file path could be accepted either as one string of
\lstinline+"/"+-separated strings, or as a list of strings).

In statically typed languages, and more specifically in functional programming
languages, unions are usually implemented as algebraic data types [references],
also called \emph{tagged unions}. Well-known programming languages with
first-class tagged unions include Haskell, OCaml, Rust, Scala, PureScript, Elm,
Swift and surely many more.

\subsection{Tagged versus untagged}
\color{red}Goals of the section

Interpreted languages already need to tag their values, hence bare union are more natural.
It also corresponds better to prior usage in untyped code\vspace{0.5cm}\color{black}

In the statically typed setting, the actual representation of a tagged union is
an implementation detail: from the point of view of the programmer, a tagged
union is a fist-class data type associated with constructors and destructors
with well-defined semantics. In contrast, in a gradually typed language like
Nickel, the question of the representation of tagged unions as untyped values
arises.

Unfortunately, tagged unions do not have a serializable canonical counterpart as
untyped data. One usual representation is a record with a tag and a value whose
shape depends on the tag (see Figure \ref{fig:union-encoding}).

\begin{figure}
  \begin{center}
    \begin{lstlisting}
        { tag: <Integer, String>, value: Dyn }
    \end{lstlisting}
  \end{center}
\caption{Encoding of the union type: integer or a string}
\label{fig:union-encoding}
\medskip
\small
In Nickel, \lstinline+<A, B, C>+ is the type of a C-like enumeration that can
take value \lstinline+A+, \lstinline+B+ or \lstinline+C+, while \lstinline+Dyn+
is the dynamic unitype.
\end{figure}

A better way to describe it would be as a dependent pair which first component
is a tag and second component is the data type corresponding to this tag:

\[
    \Sigma_{tag : <A, B, C>} \text{Alternative}(\text{tag})
\]

Nickel does not feature dependent types, which would open a whole new level of
complexity. Alas, choosing the non dependent representation as a default is not
fully satisfying.  First, it does not map precisely to a JSON value: in
consequence, a lot of illegal combinations are structurally valid as untyped
JSON values. But this is to be expected of any representation. More importantly,
this does not necessarily match with the configuration schemas one can find in
the wild, that favor simple, untagged union.

In dynamically typed languages, it is customary to directly store different kind
of data in one variable without further ceremony. The runtime system needs to
attach type tags to values anyway, which can usually be observed using functions
like JavaScript's \verb+typeof+. Using so-called \emph{untagged unions} has
already been an idiom in dynamically typed language for a long time
[ref/examples], and being able to type and handle these unions is an important
aspect of the migration to a gradual type system. Indeed, a lot of gradually
typed languages either existed originally as dynamically typed language or aim
at backward compatibility with a dynamically typed base language. This is why
they tend to favor supporting untagged union types, the typed counterpart of
this practice (as e.g in TypeScript \cite{TypeScriptUnions}).

In the case of Nickel, there is no pre-existing untyped language that we have to
support, so we are freer of our choice. However, Nickel is a configuration
language, whose contracts and types system is intended to be expressive enough
to encode data schemas encountered in practice. Since these schemas do use
untagged unions, there is a strong incentive to support untagged unions at the
type level, or at least, at the contract level.

\subsection{Tagged as untagged}
\color{red}Goals of the section

Untagged unions, modulo a slightly smarter typechecker, can encode usual type-safe
tagged unions. They let the user free of the representation though, and encode
more stuff, that's why they are appealing.\vspace{0.5cm}\color{black}

Untagged union, together with the other existing features of the type system,
are sufficient to implement the encoding illustrated in Figure
\ref{fig:union-encoding}. For example, let us represent an algebraic data type
(written in ML syntax) to an imaginary Nickel union type:

\begin{lstlisting}
type either = Left of int | Right of string
\end{lstlisting}

\begin{lstlisting}
{ tag : <Left>, val : Num} | { tag : <Right>, val : Str}
\end{lstlisting}

There is a twist, though. As it is, the static type system is unable to refine
the type of the value, making the following example rejected:

\begin{lstlisting}
switch x.tag {
  Left => x.val + 1,
  Right => strings.fromNum x.val
}
\end{lstlisting}

Flow typing, implemented in Racket\cite{FlowTypingRacket1, FlowTypingRacket2},
Groovy\cite{FlowTypingGroovy}, Whiley\cite{FlowTypingWhiley} or
TypeScript\cite{FlowTypingTypeScript}, is capable of refining appropriately the
type information in each branch. Thus equipped with untagged unions and a
typechecker smart enough, classic tagged unions can be simulated, but with the
additional benefit that the language is not imposing any representation.
Untagged unions empower the programmer to represent and handle a wider variety
of union idioms appearing in the wild.  Thus, adding union types to Nickel
sounds appealing.

\subsection{Union and intersections}
\color{red}Goals of the section

Explain why having unions probably unavoidably leads one to have intersection.\vspace{0.5cm}\color{black}

$(A -> B) /\ (C -> D) ~ (A \/ C) -> (B \/ D)$
$(A \/ B) -> C ~ (A -> C) /\ (B -> C)$

% TODO: lazyness + parallelisation

\printbibliography

\end{document}
