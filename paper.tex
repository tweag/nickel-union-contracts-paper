\documentclass{article}
\usepackage{mathtools,amssymb,amsmath,amsfonts,amsxtra,amsthm,mathrsfs,mathpartir}
\usepackage{xcolor}
\usepackage{array}
\usepackage[toc,page]{appendix}
\usepackage{graphicx}
\usepackage{varioref}
\usepackage{verbatim,comment}
\usepackage{longtable}
\usepackage{tikz}

\title{Union and intersection contracts: theory vs practice}
\author{Teodoro Freund, Yann Hamdaoui, Arnaud Spiwack}

\begin{document}

\maketitle

\begin{abstract}
 In this paper we do some cool stuff.
\end{abstract}

\section*{Introduction}
Intro

\section{Background on union types}

\subsection{Nickel design space}

Make clear the design of Nickel aims at being simple, interoperable with JSON,
and to not type every possible untyped idioms, but rather the common typed one...

\subsection{Tagged versus untagged}

Unions are fundamental and ubiquitous pieces of programming. While products
represent conjunctions, which are several pieces of data put together, unions
represent disjunctions, that are different alternatives values for the same
data. Put together, they allow to model a large class of data types. 

\\
EXAMPLES of union in configuration specification
\\

In statically typed languages, more specifically in functional programming
languages, they are usually represented as algebraic data types [references],
also called *tagged unions*. Tagged means that there is an additional field, or
constructor, that allows to discriminate between each alternatives, usually at
compile-time. Well-known programming languages with first-class tagged unions
include Haskell, OCaml, Rust, Scala, PureScript, Elm, Swift and a even greater
number of related and derived languages.

\\
EXAMPLES
\\

All of the aforementioned languages are statically typed. In this setting, the
actual representation of a tagged union is an implementation detail: from the
point of view of the programmer, a tagged union is a proper value 
associated with a clearly defined semantics. However, in a gradually typed language like Nickel,
the question of how tagged union are represented as untyped values arises. One
problem with tagged union is that it is a fundamentally typed representation,
which has no totally natural counterpart as an untyped data.

One usual representation is as a structure with a tag and a value which actual
shape depends on the tag:

\\
FIGURE OR C CODE EXAMPLE ILLUSTRATING THE USUAL REPRESENTATION
\\

But this is neither totally natural, nor the only possible solution: 
. More importantly, this is not a natural representation for schemas, which
favors simple, untagged union. Indeed, if one can determine the type at
run-time, and there is no typy stuffy, the tag sounds rather useless.

In an untyped language, it is thus natural to directly stores different data in a
variable: actually, the runtime system is the one in charge of tags, which it
makes accessible via functions such as JavaScript's
\begin{verbabtim}typeof\end{verbatim}. Using so-called untagged unions has already
been an idiom in practice for a long time, and being able to type and represent
these unions is an important aspect of a type system applying to these
languages.

This is the reason why other gradually typed languages, which were either
originally untyped or aim at backward compatibility with an untyped base
language, seem to favor supporting union types which are the typed counter-part
of these untagged union [example of gradually typed language with untagged
union].

In the case of Nickel, there is no pre-existing untyped language that we have to
support, so we are freer of our choice. But are we? As a configuration language,
Nickel does not evolve in the abstract. It aims at maximum interoperability with
JSON. It is designed to generate data for other software to consume, and strives
to be able to encode as much schemas as possible inside the language itself. If
these schemas use untagged union, then this is a strong incentive to support
them at the type level, or at least, at the contract level.

\subsection{Union types}

\subsection{Tagged as untagged}

Note that together with the other existing features of the type system, one can
encode tagged union in the following way:

\\
Encoding as: use singleton contract for tags, and then union.
\begin{lstlisting}
type sum = A of Num | B of Str
\end{lstlisting}

\begin{lstlisting}
let Singleton = fun val l x => if val == x then x else blame l in
let A = Singleton `A in
let B = Singleton `B in
{ tag : #A, val : Num} | { tag : #B, val : Str} 
\end{lstlisting}

This is just the encoding mentioned in the previous section. But it suffers a
fundamental problem: the static type system is not able to deduce when the value
is known to be a `Num` or a `Str`.

\\
EXAMPLE if x.tag == `A then x.val + 1 else if val.tag == `B then strings.fromNum
x.val
\\

TypeScript is capable of emulating such union (see ??).

\end{document}
