\newcommand{\moral}[1]{\noindent $\hookrightarrow$ \textsc{#1}}

Type systems featuring intersection types are hard in a fundamental way,
from their discovery, they've been studied as a way to characterize
lambda-calculus terms normalizability; which means that typechecking a
term on a system with intersection types is undecidable,
similar to how the halting problem is undecidable\todo{reference}.

Usually, when thinking about programming languages, there's an intuition
that static checks (like typechecking) are more difficult and less flexible
than dynamic checks (like testing). For instance, it is very easy to check if,
in a given execution of a program, a variable holds an integer value; while it
may become more challenging to check if a variable will always hold an integer value,
no matter the particular execution.

As we intend to show on this paper, union and intersection types are not only
statically hard to check, but they are also dynamically hard;
\emph{they are fundamentally hard}.
In this section, we present intuitively the main challenges that arise
when attempting to develop a runtime check system for union and intersection types.

\subsection{Known challenges}

There are two works proposing solutions to dynamically checking union and intersection
types (through contracts)\todo{reference}. 
We briefly review the main challenges they consider, and some extra ones.
At the same time, we outline the reasons that make these works,
in our opinion, non satisfactory.

\todo{add Castagna "Gradual types a new perspective"}

\subsubsection*{Union contracts may need multiple evaluations}

When working with contracts, it is assumed that an error may not be raised, even
when one may exist; contracts allow false negatives.\change{Unclear}

% Maybe a better example is not using function contracts, since these are "special"
% For instance, it could be to have $(if (randomBool()) then 42 else "hello")@String$
% There is an error in that expression, but you may not find it

For instance, imagine you have a function $g$, that
you wrap with a contract asserting that it sends positive values to positive values,
($let~f~=~g@Pos \rightarrow Pos$).
A first call $f~10$ may return $3$, complying with the contract.
While a second call, $f~5$, may return $-2$, raising an error since
the contract has been violated.
Even if only after evaluating both of this function calls
the error was found, the first one is not really needed, and $f~5$ is enough to
prove that $f$ doesn't comply with the given contract.

When unions get involved, this desired property is lost,
\change{What property?}
for instance, imagine having the function $let~g~=~\lambda~x.~if~x~then~2~else~"Hello"$,
and wrapping it with the contract $(Bool \rightarrow Bool) \cup (Bool \rightarrow Num)$
(i.e. $let~f~=~g@(Bool \rightarrow Bool) \cup (Bool \rightarrow Num)$),
it's clear that $g$ doesn't comply with either of those function contracts, however, there
is not a single call to $f$ that would raise, we need the combination of two calls,
$f~True$ and $f~False$ to find the violation; and the partial
information that these calls failed cannot be forgotten.
\info{IMPORTANT PROPERTY: minsize(witness) > 1}

\moral{Union contracts behave globally}

This seemingly simple consequence breaks referential transparency on
contract wrapping in functional languages, that is
$let~h~=~g@(Bool \rightarrow Bool)~\cup~(Bool \rightarrow Num)~in~(h~True,~h~False)$
won't behave the same as 
$(g@(Bool \rightarrow Bool) \cup (Bool \rightarrow Num) True, g@(Bool \rightarrow Bool) \cup (Bool \rightarrow Num) False)$.

\todo{Expand on the last paragraph}

\subsubsection*{Intersection contracts must not share information}

Dually, intersection types need the contrary property\change{Same: what
property?}.
Imagine having a function, with some unimportant implementation,
and wrapping it
with a contract stating that it behaves both as a $Bool$ to $Bool$ function,
and as a $Num$ to $Num$ function; that is, 
$let~f~=~...@(Bool \rightarrow Bool) \cap (Num \rightarrow Num)~in~(f~42,~f~"hello")$.
Now, the first application ($f~42$) invalidates the first contract, but complies with the second
one by applying a number; the second application ($f~"hello"$), behaves dually,
it invalidates the second contract,
while complying with the first one.

This example shouldn't fail, even if the whole program invalidates both choices of $f$, it does
so at two different points of the execution and it should not raise an exception.
Therefore, the elimination of an intersection contract must be local to each elimination
context, as pointed out by Keil and Thiemann \cite{KeilThiemannUnionIntersection}.

\moral{Intersection contracts behave locally}\\

\info{Maybe another way of saying it is each instance must be executed in a
separate environment}
% GOAL: State that intersection and union, without HO, is easier. And that is being implemented on nickel.

As one dives into the problems that may become present when dealing with union and intersection,
a valid question is how these could be implemented in the simplest of cases.
For instance, imagine having
union and intersection only present for lower ordered types 
($Bool$, $Num$, $String$, $(Num, Bool)$, etc.), then checking
if a particular value is a $String$ or a pair of $Num$ and $Bool$ is easy,
just check both and fail only if both checks fail.
Checking if a value complies with an intersection of two contracts, $A$ and $B$ is even
easier, just apply both contracts independently.

Typed Racket implements this idea, but calls intersection and union contracts
\texttt{and} and \texttt{or}, respectively.

\moral{It is the combination of union, intersection, and function contracts that is not trivial}
\change{The moral is true, but maybe restate the problem are functions. Do we
    want to say that since Nickel is lazy, almost every value is morally a
function (thunk)?}

\subsubsection*{Flat contracts are not idempotent}

Following the simple case presented above, there's another kind of extension that could be
applied to a contract system, and that is contracts written by the
programmer to check arbitrary properties; these are usually called flat contracts and
involve a predicate over the contracted value that returns a boolean, indicating whether
the contract passed or it should raise an exception.

For instance, one may build a contract that checks that the value in question is a
function that, when applied to $1$, returns $1$, as something like
$\lambda f.~f~1~==~1$; let us call it $C$.
Then, wrapping a given value to check that it complies with this contract, but that at
the same time is a valid $String \rightarrow String$ function may look something like this
$let~f~=~...@C \cap (String \rightarrow String)$, a valid implementation for this function
may be $\lambda x.~x$.
However, depending on how this intersection is resolved, we may face the problem of applying first
the $String$ to $String$ contract, and then $C$, in which case we would try to apply a function,
wrapped with a $String \rightarrow String$ contract, to 1, raising blame on the context
of the contract execution.

To solve this problem, each execution of a contract has to be aware of its execution context,
or of its precedence.


% TODO come up with an example not using HO contracts, we need to present Nickel's flat contracts,
% that are much more permissive

\moral{The addition of flat contracts involves evaluation context dependency.}

Of course, one could ask himself whether applying these two contracts separately,
could make any sense, but this bring either some kind of exponential explosion
on the execution, or a need for a non deterministic execution
strategy.
The authors of this papers spent some time trying to solve this by using
\texttt{call/cc}, an attempt that became too unpractical too rapidly.

Keil and Thiemann propose a non deterministic strategy for their contract checking
calculus; this gives a very simple and elegant solution, but, sadly, still
unfeasible to implement.

They also propose a different approach, by checking the context at which
each contract is opened, dropping
it if the context indicates it is being applied inside the evaluation of another
(related by an intersection) contract.
This check implies traversing the context of execution.
We consider that having execution dependent on the context
of evaluation breaks locality of
the language execution in undesired ways.\todo{Unclear. Is it a bad thing
    "philosophically" (in general, because it breaks local reasoning?) If there
are concrete unwiding examples, we should add some}.

WMW decided to disallow flat contracts completely,
we believe this hurts badly a dynamically checked framework, since it goes against
the common practice, in dynamic languages, of checking for properties
that go beyond the expressivity given by a type system.
\todo{Maybe mention explicitly the word "validation": contracts allow for
    principled, higher-order compatible, composable validation. For validation,
you want to check things much more complex than just belonging to a type.}

\subsubsection*{Inflexibility of union and intersection contracts}

Solutions for union and intersection contracts do exist (KT, WMW) however,
they do not provide the flexibility that one may be used to when using dynamic checks.
Consider a Python programmer, she would not care to see something like this
\texttt{assert(isinstance(3 if True else "hello", int)))}, while a Haskell programmer
might perform a harakiri upon reading it.

Without falling into the eternal discussion of static vs dynamic type checking,
regular users of dynamically checked languages expect this flexibility.
Sadly, the current work on union and intersection does not provide it.

For instance, consider the following code
% \texttt{(\\f.(\\xy.x) (f 1) (f true))((\\xy.x)@(N -> N -> N) /\\ (N -> B -> N) 1)}.
$let~f~=~((\lambda~x~y.x)@(N -> N -> N) \cap (N -> B -> N))~1~in~(f~1,~f~True)$.
If we use any of the KT or WMW frameworks, both our Haskell programmer and our
Python programmer will change their career path towards photography, or something
similar.
Our Haskell programmer, will do so because this code compiles;
while the Python programmer would do so because this code fails on execution.

The reason this happens, is that even if you would expect
$(N -> N -> N) \cap (N -> B -> N)$
and $N -> (N -> N) \cap (B -> N)$ to behave equally, the point at which
the decission of which path of the intersection to use is taken
matters.
When working with dynamic checks, it is expected that this kind of
flexibility works.

\todo{better phrasing and try to use the dual example}

\moral{Are union and intersection contracts really what we want? Is there an alternative (simpler) option?}

\subsection{Complexity of existing solutions}

The Keil and Thiemann paper presents a complete system that handles union and intersection
contracts in the presence of higher order contracts, however, we consider it to be overly
complicated for a reasonable to implement language.
In what follows, we state some of the reasons that we consider troublesome.

% \subsubsection{Non deterministic semantics}

% The semantics of the calculus are presented, primarly, on a non deterministic interpretation,
% which means that, even if their work provides a deterministic semantics,
% the expected way to understand
% the behavior is not the same as the expected way to implement it, cluttering the development
% of a language that could provide this solution.

% Keil and Thiemann had also developed a contract system for JavaScript, called TreatJS, that
% implements their ideas, but it does so with a different technique that the one presented
% on the paper.

% On our opinion, having a formalization that lies far away from the implementation
% hearts any future expansion work on the system, and it is not a satisfactory solution
% for what we intended to pursuit.


% \todo{Maybe mention call/cc}

% \subsubsection{Context opening}

% Keil and Thiemann noticed that flat contracts cannot be arbitrarily used, since
% violating a flat contract inside another contract would, unexpectedly, raise blame.
% Their solution, is to check the context at which each contract is opened, dropping
% it if the context indicates it is being opened inside the evaluation of another
% contract.

% This check, implies traversing the context of execution.
% We consider that having execution dependent on the context
% of evaluation breaks locality of
% the language execution in undesired ways.

\subsubsection{Contract normalization}

GOAL: The relation between union and intersection have to be modified, transforming them
in a union of intersections.

In the Keil and Thiemann paper, before the validation of contracts can begin, the
contracts have to be normalized into unions of intersections of (other) contracts.
This problem  is the starting point from which
WMW deliver their work, and we share their opinion:

"However, the monitoring semantics for contracts of intersection and union types given by Keil
and Thiemann are not uniform. (...) If uniformity helps composition, then
special cases can hinder composition.", [Ref]

\subsection{Shortcomings in WMW}

The second paper worth mentioning is the one by Williams, Morrison and Wadler [ref].
This work, attempts to provide a simpler contract system with
union and intersection types.
However, we believe this work has some issues and shortcomings, which make it
not that useful in practice; we briefly outline them below.

% \subsubsection*{Lack of flat contracts}

% In order to save themselves from dropping some contracts when checking a union
% or intersection branch, WMW decided to eliminate completely flat contracts from
% their system.
% We believe this hurts badly a dynamically checked framework, since it goes against
% the common practice, in dynamic languages, of checking for non trivial properties.

\subsubsection*{Problem in semantics (monitoring)}
Is the problem in the semantics worth mentioning? or is it attacking too much on the paper.
