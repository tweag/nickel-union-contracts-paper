Type systems featuring intersection types are hard in a fundamental way,
from their discovery, they've been studied as a way to characterize
lambda-calculus terms normalizability; which means that typechecking a
term on a system with intersection types is undecidable,
similar to how the halting problem is undecidable.

Usually, when thinking about programming languages, there's an intuition
that static checks (like typechecking) are more difficult and less flexible
than dynamic checks (like testing). For instance, it is very easy to check if,
in a given execution of a program, a variable holds an integer value; while it
may become more challenging to check if a variable will always hold an integer value,
no matter the particular execution.

As we intend to show on this paper, intersection (and union) types are not only
statically hard to check, but they are also dinamycally hard;
\textit{they are fundamentally hard}.
In this section, we present intuitively the main challenges that arise
when attempting to develop a runtime check system for intersection and
union types.

\subsection{Known challenges}

There are two works proposing solutions to dynamically checking union and intersection
types (through contracts). We briefly review the main challenges they consider.

TODO: add Castagna "Gradual types a new perspective"

\subsubsection*{Union contracts may need multiple evaluations}

When working with contracts, it is assumed that an error may not be raised, even
when one may exist. For instance, imagine you have a function \texttt{f}, that
you wrap with a contract asserting that it sends positive values to positive values.

A first call \texttt{f 10} may return \texttt{3}, complying with the contract.
While a second call, \texttt{f 5}, may return \texttt{-2}, raising an error since
the contract has been violated. Even if only after evaluating both of this function calls
the error was found, the first one is actually not needed, and \texttt{f 5} is enough to
prove that \texttt{f} doesn't comply with the given contract.

When unions get involved, this desired property of only needing one
use of the contracted object to find an error is lost,
for instance, imagine wrapping $f = \lambda~x.~if~x~then~2~else~"Hello"$
with the contract $(Bool \rightarrow Bool) \cup (Bool \rightarrow Num)$, it's
clear that $f$ doesn't comply with either of those function contracts, however, there
is not a single call to $f$ that would raise, we need the combination of two calls,
$f~True$ and $f~False$ to find the violation; and the information that these calls failed
cannot be forgotten.

\textsc{Union contracts are not local}

% TODO expand on this, or remove it

This seemingly simple consequence breaks referential transparency on
functional languages, that is~
$let~g~=~f@(Bool \rightarrow Bool)~\cup~(Bool \rightarrow Num)~in~(g~True,~g~False)$
won't behave the same as 
$(f@(Bool \rightarrow Bool) \cup (Bool \rightarrow Num) True, f@(Bool \rightarrow Bool) \cup (Bool \rightarrow Num) False)$.

\subsubsection*{Intersection contracts must not share information}

On the oppositde side, intersection types need the contrary property.
Imagine having $let~f~=~...~in~(f~42,~f~"hello")$, where the $...$ are filled
with some unimportant implementation. And then consider wrapping the function
$f$ with a contract stating that it behaves both as a $Bool$ to $Bool$ function,
and as a $Num$ to $Num$ function, that is $(Bool \rightarrow Bool) \cap (Num \rightarrow Num)$.
Now, the first application ($f~42$) invalidates the first contract, but complies with the second
one by applying a number; the second application ($f~"hello"$), behaves dually,
it invalidates the second contract,
while complying with the first one.

Thus, even if the whole program invalidates both choices of $f$, it does
so at two different points of the execution and it should not raise an exception.
Therefore, the elimination of an intersection contract must be local to each elimination
context, as pointed out by Keil and Thiemann [ref].

\textsc{Intersection contracts must be local}\\

% GOAL: State that intersection and union, without HO, is easier. And that is being implemented on nickel.

As one dives into the problems that may become present when dealing with union and intersection,
a valid question is how these could be implemented in the simplest of cases.
For instance, imagine having
union and intersection only present for base types ($Bool$, $Num$, $String$, etc.), then checking
if a particular value is a $Bool$ or a $Num$ is easy, just check both and fail only if both 
checks fail. Checking if a value complies with an intersection of two contracts, $A$ and $B$ is even
easier, just apply both contracts independently.

\textsc{It is the combination of union, intersection, and function contracts that is not trivial}

\subsubsection*{Flat contracts are not idempotent}

Following the simple case presented above, there's another kind of extension that could be
applied to it besides higher order contracts, and that is contracts written by the
programmer to check arbitrary properties; these are usually called flat contracts and
involve a predicate over the contracted value that returns a boolean, indicating whether
the contracts passed or it should raise an exception.

For instance, one may build a contract that checks that the value in question is an
even number, as something like $\lambda x.~x~mod~2~==~0$.

% TODO come up with an example not using HO contracts, we need to present Nickel's flat contracts,
% that are much more permissive

\textsc{The addition of flat contracts involves evaluation context dependency.}

\subsubsection*{Inflexibility of union and intersection contracts}

Solutions for union and intersection checks with contracts do exist (KT, WMW) however,
they do not provide the felxibility that one may be used to when using dynamic checks.
Consider a Python programmer, she would not care to see something like this
\texttt{assert(isinstance(3 if True else "hello", int)))}, while a Haskell programmer
might perform a harakiri upon reading it.

Without falling into the eternal discussion of <insert static vs dynamic>,
regular users of dynamically checked languages expect this flexibility.
Sadly, the current work on union and intersection does not provide it.

For instance, consider the following code
\texttt{(\\f.(\\xy.x) (f 1) (f true))((\\xy.x)@(N -> N -> N) /\\ (N -> B -> N) 1)}.
If we use any of the KT or WMW frameworks, both our Haskell programmer and our
Python programmer will change their career path towards photography, or something
similar.
Our Haskell programmer, will do so because this code would compile;
while the Python programmer would do so because this code fails on execution.

TODO: keep explaining and show the dual example



\subsection{Impracticallity of KT}

The Keil and Thiemann paper presents a complete system that handles union and intersection
contracts in the presence of higher order contracts, however, we consider it to be overly
complicated for a reasonable to implement language.
In what follows, we state some of the reasons that we consider troublesome.

\subsubsection{Non deterministic semantics}

The semantics of the calculus are presented, primarly, on a non deterministic interpretation,
which means that, even if their work provides a deterministic semantics,
the expected way to understand
the behavior is not the same as the expected way to implement it, cluttering the development
of a language that could provide this solution.

Keil and Thiemann had also developed a contract system for JavaScript, called TreatJS, that
implements their ideas, but it does so with a different technique that the one presented
on the paper.

On our opinion, having a formalization that lies far away from the implementation
hearts any future expansion work on the system, and it is not a satisfactory solution
for what we intended to pursuit.


TODO Maybe mention call/cc

\subsubsection{Non local (context opening) operation}

% TODO rewrite
Keil and Thiemann noticed that flat contracts cannot be arbitrarily used, since
violating a flat contract inside another contract would, unexpectedly, raise blame.
Their solution, is to check the context at which each contract is opened, dropping
it if the context indicates it is being opened inside the evaluation of another
contract.
This check, implies traversing the context of execution. We believe that
having execution dependent on the context of evaluation breaks locality of
the language execution in undesired ways.

\subsubsection*{Unions and intersections have to be opened up}

GOAL: The relation between union and intersection have to be modified, transforming them
in a union of intersections.

In the KT paper, before the validation of contracts can begin, the types have to bw
transformed, into unions of intersections of (other) contracts. This problem,
is the starting point from which WMW deliver their work, and we share their opinion:

"However, the monitoring semantics for contracts of intersection and union types given by Keil
and Thiemann are not uniform. (...) If uniformity helps composition, then
special cases can hinder composition.", [Ref]

\subsection{Shortcomings in WMW}

The second paper worth mentioning is the one by Williams, Morrison and Wadler [ref].
This work, attempts to provide a simpler contract system with
union and intersection types.
However, we believe this work has some issues and shortcomings, which make it
not that useful in practice; we briefly outline them below.

\subsubsection*{Lack of flat contracts}

In order to save themselves from dropping some contracts when checking a union
or intersection branch, WMW decided to eliminate completely flat contracts from
their system.
We believe this hurts badly a dynamically checked framework, since it goes against
the common practice, in dynamic languages, of checking for non trivial properties.

\subsubsection*{Problem in semantics (monitoring)}
Is the problem in the semantics worth mentioning? or is it attacking too much on the paper.
